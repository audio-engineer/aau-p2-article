% textidote: ignore begin
\chapter{Introduction}\label{ch:introduction}
% textidote: ignore end

Learning is something we do every day.
It is something most people enjoy and strive to experience.
However, learning can also be difficult.
In this report we will explore the concept of collaborated assisted learning and how software can assist in this method.
We will be focusing our analysis around the age-old game of chess.

Chess has seen a rise in player count in recent years, and with the growing player base the current chess websites have
perfected their craft, this has been done through intuitive design and the time to implement extra features in the apps.
One reason that chess has become more popular online, is that using the popular web apps like \url{chess.com} or
\url{lichess.org} allows the player to use a powerful tool to aid in their learning journey, that is chess engines.

Chess engine popularity has risen greatly in the past years and have been developed to a point that they
outperform humans.
These tools are essential to understanding the complexity of the end- and mid-game strategies of chess.
Chess encapsulates the spirit of a game that is easy to learn, but hard to master.
We will be focusing on the learning aspect of chess by making a chess web app, that allows two players to play against
each other with a coach.
This coach will be a chess engine, which will allow both players to learn while playing in a multiplayer environment.

Through this approach we touch a niche that lies between the social aspect of a player vs player chess game and the
great learning tools that are made to study chess and its complexities individually.
The product we are suggesting would then take the best of both worlds, by targeting the social aspects of the game while
still providing useful information to the player on how to improve their chess capabilities.
We go into more detail in Section~\ref{sec:state-of-the-art}.
