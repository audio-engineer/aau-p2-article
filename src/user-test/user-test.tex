\chapter{User test}\label{ch:user-test}

After the development of the website and the game, the team conducted a user test to gather feedback on the project.
That feedback would be used to improve the website and game before the final release.
It could also provide the team with insights into the user experience and the game's playability.

\section{Methodology}\label{sec:methodology}

The user tests were conducted by multiple members of the team, so to keep the results consistent, a methodology was
established.
It consisted of a script that the team members followed, ensuring that the same questions were asked to each
participant.
The script was divided into four parts: introduction, user interface and experience, gameplay, and feedback.
The parts were designed like an interview, where the team members would first show them the aspects of the website,
then let them play the game for themselves, and finally ask a series of questions to gather their feedback.
The results in Section~\ref{sec:results} are based on their feedback rather than the script.

\section{Participants}\label{sec:participants}

A total of three participants were recruited for the user test.
They varied between age and skill level, with at least two of them fitting the target audience of the game.
This section will introduce the participants and their background.

\begin{itemize}

    \item \textbf{Ronja} is 22 years old, and she ranks at 800 ELO, which fits in the target audience of beginners.
    She was taught to play in her kindergarten, and now she plays chess occasionally for fun.

    \item \textbf{Clara} is 19 years old, and she ranks at 1000 ELO, which also fits in the target audience.
    She studies machine learning, so she could provide valuable feedback on the Stockfish implementation.
    She was taught to play chess in an academic setting.

    \item \textbf{Peter} is 31 years old, and his ranks is 1700 ELO, which is above the target audience.
    He plays chess semi-professionally, first learning to play from his father and then in school.
    % textidote: ignore begin % due to chess.com
    He got more interested in chess during the pandemic due to Chess.com's popularity and chess' presence on Twitch.
    % textidote: ignore end
    While he may not be the target audience, his experience should provide the team with valuable feedback.

\end{itemize}

% textidote: ignore begin % due to section being too short
\section{Results}\label{sec:results}
% textidote: ignore end

The results of the user test are based on the feedback from the participants.
The feedback was divided into three categories similar to the methodology: interface and experience, gameplay, and
feedback.
Each of these categories will have a mixture of all participants' feedback, providing a comprehensive view of the user
test.

\subsection{Interface and experience}\label{subsec:ui/ux}

The overall feedback on the interface was positive.
Everyone thought that the website was visually appealing, calling it pretty.
The team did however receive some constructive criticism.

Peter pointed out that the front page was not intuitive and Clara suggested that the buttons should be larger, labelled,
and link together.
Peter also mentioned that it was difficult to figure out how to play, he proposed that the About page should contain
information on how to create an account and what the website is about.

The in-game interface was also critiqued.
Clara noted that the Stockfish feedback lacked visual cues, her idea was that there should be arrows on the chessboard,
which is something that the team had previously discussed.
Peter mentioned that the notations provided by Stockfish were incorrect and complained about the lack of sound effects
and there being no apparent indication for when a game is over.
Meanwhile, Ronja thought that nothing was missing.

\subsection{Gameplay}\label{subsec:gameplay}

The users critiqued the gameplay, with the main focus being on the Stockfish implementation.
Clara thought that the Stockfish feedback was too direct, suggesting that it should give smaller hints instead of the
direct solution.
Peter mentioned that the feedback should provide future moves along with the current move.
Ronja didn't immediately understand what the Stockfish feedback was, but after using Stockfish a little, she lost
interest in it.
This indicates that the current Stockfish implementation has a lot of room for improvement on what feedback it provides
and how to provide it to the player.

Clara also mentioned that the game should have more teaching tools, such as puzzles and guides to how to play chess.
The team considered these features given the learning nature of the game, but there are already existing solutions that
provide them, so they were not prioritized, instead focusing on the coaching aspect of the game.

Lastly, the users were asked about the idea of using machine learning along with Stockfish to provide feedback.
The rationale behind this question was due to a prior discussion within the team, which will be later discussed in
Section~\ref{sec:large-language-model-implementation}.
Clara and Ronja both liked the idea of using machine learning, while Peter was more skeptical.
He did not particularly like the idea due to the lack of control over the large language model's answers, so the team
cannot be guaranteed that it'll provide correct information.
Clara however thinks that in combination with Stockfish it could provide accurate information whilst also giving more
natural responses and explanations.

% textidote: ignore begin % due to section being too short
\subsection{Feedback}\label{subsec:feedback}
% textidote: ignore end

To conclude the user test, the participants were asked to provide their final thoughts on the project, from the idea to
the execution.
Peter's thoughts are a little mixed.
Ronja had a positive experience, but only after she understood how to use Stockfish.
Clara thinks that the website can motivate beginners to play chess, and she wishes that she used an engine like Stockfish
at the academy where she learned to play chess.
% textidote: ignore begin % due to chess.com
She and Ronja would use the website again, but Peter would rather use alternatives like Chess.com and Lichess, because
he believes that they can do everything our website can do and more.
% textidote: ignore end

% textidote: ignore begin % due to section being too short
\section{Conclusion}\label{sec:tests-conclusion}
% textidote: ignore end

The feedback from the users was generally positive, but there are still areas that need improvement.
The team received a lot of suggestions for areas to improve, some of which were already discussed prior to the user
tests, but also some new ones.
For example, expanding the About page to include information on how to create an account and what the website is about.
The team already knew what it's about, but had not considered that new users might not.
Another example is the wrong notations provided by Stockfish, which the team was not aware of.
Feedback on the Stockfish implementation was also valuable, but some suggestions could not be implemented due to time
constraints.
Overall the user test was a success, and the feedback provided valuable insights for the team on how to improve the
website and game before the final release.
