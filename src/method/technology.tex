\section{Technology}\label{sec:technology}

Throughout the development of the project, a number of different tools and publicly available packages are used to
assist the development process.
These different technologies were chosen based on the team's needs and the requirements of the project.
The following sections will describe the different technologies used in the project.

\subsection{Programming stack}\label{subsec:programming}

As the project is a web application, TypeScript~\cite{typescript} is used as the main programming language.
TypeScript is a superset of JavaScript, with the main difference being that TypeScript is statically typed, while
JavaScript is dynamically typed.
It can be very useful in larger projects, as it helps to catch errors during compile-time as opposed to run-time.
This is the first time for many of the team members to use TypeScript, but they find it to be safer than JavaScript
for larger projects.

The frontend of the project is built with React~\cite{react}, which is a JavaScript / TypeScript library for building
user interfaces.
It is developed and maintained by Meta, and is used by many large companies.
React is focused on mobile applications first and foremost, but the team's focus was on the desktop experience.

% textidote: ignore begin % due to node.js and next.js being identified as an url
The backend is built using Node.js~\cite{node.js}, which allows the team to build server-side components using
TypeScript.
Node.js is a popular choice for backend development, as it is fast and scalable.
As well as using React for client-side rendering, Next.js~\cite{next.js} is incorporated to handle the server-side
rendering.
Next.js is a React framework running on top of Node.js, which assists in building the frontend and backend
together.
% textidote: ignore end

Because Stockfish~\cite{stockfish} is written in C++, the team created a REST API using Python to communicate with it.
Python was chosen, as it already has a wrapper called py-stockfish~\cite{py-stockfish}, which assists in the development
of the API\@.
The API has functions that take input from the server, run it through Stockfish, and return the output back to the
server.
Unit tests are used to ensure that the API's functions are working correctly.

\subsection{JavaScript libraries}\label{subsec:libraries}

Throughout the project the team uses a number of different libraries to aid with the mundane tasks of web development.
% textidote: ignore begin % Due to chess.js being identified as an url
For example, as the focus was on the learning aspect of the project, the team uses Chess.js~\cite{chess.js} to handle
the chess logic and react-chessboard~\cite{react-chessboard} to display the chessboard.
% textidote: ignore end
This saves a lot of time, as chess is a complex game with many rules.

The team also uses Material UI~\cite{mui} (MUI) to style the website, as it is a modern and popular library that
uses Google's design language, Material Design.
MUI is used to create the buttons, text fields, and other components that make up the website.

Lastly, the team uses Chatscope~\cite{chatscope} to create the chat window of the website.
The chat window is an important element of the website, as it allows users to communicate with Stockfish or with each
other.

\subsection{Hosting provider}\label{subsec:hosting}

There are a number of different hosting providers available.
Amazon Web Services~\cite{aws} is a popular choice, but it can be expensive.
Instead, Firebase~\cite{firebase} was chosen due to its features and pricing.
The main reason for this decision is that Firebase allows for easy authentication with Google accounts, which was
essential for building the lobby system.

% textidote: ignore begin % Due to socket.io being identified as an url
Socket.io~\cite{socket.io} was considered for the multiplayer aspect, but it can't be used with Firebase.
% textidote: ignore end
However, Firebase provides a real-time database, which was extensively used throughout the project instead.
The real-time database works like a normal database, but with the added benefit of being able to listen for changes in
real time and having very little delay between the data being sent and received.
% textidote: ignore begin % Due to socket.io being identified as an url
It was used as a replacement for Socket.io, as the latency was low enough for the team's needs.
% textidote: ignore end
The database saves game states, chat messages, lobbies and other data.

\subsection{Development environment}\label{subsec:development}

The project requires a number of different services to be running at the same time, which can be difficult to manage.
% textidote: ignore begin % Due to node.js being identified as an url
That is the Node.js backend, Stockfish and Firebase Emulator.
% textidote: ignore end
To manage them, the team uses Docker~\cite{docker} to simultaneously run all the services.
Docker is a platform that allows developers to isolate their applications in containers, which are lightweight and can
run on any machine.
This makes it painless to develop multiservice projects, as there are no compatibility issues between the different
devices and services.
To handle multiple containers, the team uses Docker Compose, which is a tool for defining and running multi-container
Docker applications.

As the previous project, the team uses GitHub~\cite{github} for version control.
GitHub provides many features, such as pull requests, issues, and actions, which makes it easy to collaborate and work
on the project.
The team made use of GitHub projects to plan out different tasks, issues to track bugs and features, and pull requests
to review and merge code.
GitHub Actions were also used to test or compile the program automatically.
