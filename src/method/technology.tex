\section{Technology}\label{sec:technology}

Throughout the development of the project, a number of different tools and publicly available packages are used to
assist the development process.
These different technologies were chosen based on the team's needs and the requirements of the project.
The following sections will describe the different technologies used in the project.

\subsection{Programming stack}\label{subsec:programming}

As the project is a web application, TypeScript~\cite{typescript} is used as the main programming language.
TypeScript is a superset of JavaScript, with the main difference being that TypeScript is statically typed, while
JavaScript is dynamically typed.
It can be very useful in larger projects, as it helps to catch errors during compile-time as opposed to run-time.
This is the first time for many of the team members to use TypeScript, but they find it to be safer than JavaScript
for larger projects.

The frontend of the project is built with React~\cite{react}, which is a JavaScript/TypeScript library for building
user interfaces.
It is developed and maintained by Meta, and is used by many large companies.
React is focused on mobile applications first and foremost, however it is also useful for desktop applications, why it
is chosen as the top level UI library.

% textidote: ignore begin % due to node.js and next.js being identified as an url
The backend is built using Node.js~\cite{node.js}, which allows the team to build server-side components using
TypeScript.
Node.js is a popular choice for backend development, as it is fast and scalable.
As well as using React for client-side rendering, Next.js~\cite{next.js} is incorporated to handle the server-side
rendering.
Next.js is a React framework running on top of Node.js, which assists in building the frontend and backend
together.
% textidote: ignore end

\subsection{JavaScript libraries}\label{subsec:libraries}

Throughout the project the team uses a number of different libraries to aid with the mundane tasks of web development.
% textidote: ignore begin % Due to chess.js being identified as an url
For example, as the focus was on the learning aspect of the project, the team uses Chess.js~\cite{chess.js} to handle
the chess logic and react-chessboard~\cite{react-chessboard} to display the chessboard.
% textidote: ignore end
This saves a lot of time, as chess is a complex game with many rules.

The team also uses Material UI~\cite{mui} (MUI) to style the website, as it is a modern and popular library that
uses Google's design language, Material Design.
MUI is used to create the buttons, text fields, and other components that make up the website.

Lastly, the team uses Chatscope~\cite{chatscope} to create the chat window of the website.
The chat window is an important element of the website, as it allows users to communicate with Stockfish or with each
other.

% textidote: ignore begin

\subsection{Hosting}\label{subsec:hosting}
% textidote: ignore end

Since chess, as previously discussed, is inherently a multiplayer game,
infrastructure is required that accommodates this requirement.
More specifically, online multiplayer games, as the name implies, require some kind of network connection between the
players, which is usually facilitated by the internet.
Therefore, it was obvious that the solution web application would have to be hosted on the internet.
To complicate issues further, it was determined that the web application would have to consist of multiple separate and
independent components, that would all have to be hosted online and communicate with each other over the internet.
Those components include: The front end application, the authentication and match databases, and the Stockfish-powered
processing and analysis application.

There are a number of different hosting providers available that were considered for the front end application.
Amazon Web Services~\cite{aws} is a popular choice, but can be expensive and requires a large time investment.
Instead, Vercel~\cite{vercel} was chosen due to its features, convenience and low pricing.
Furthermore, Vercel is also the vendor that makes \url{Next.js}, the front end framework that was selected,
so it was also convenient to pick a hosting provider that features out-of-the-box support for the framework in use.

As for the authentication and database, Firebase~\cite{firebase} was considered and eventually settled on.
The main reason for this decision is that Firebase allows for easy authentication using Google accounts, which was
convenient to use and essential for building the multiplayer system that requires some form of authentication,
i.e., unique player identities.
Additionally, Firebase features a real time database service, which is a NoSQL database and WebSocket server in one.
Such a service is a crucial building stone for a real time multiplayer application like our proposed solution, because
it facilitates real time exchange of state and other data between the client front end applications.
To conclude, Firebase was chosen because it provides a lot of the features that are required to create an application
like the solution in question.

And lastly, the Stockfish service needed to be hosted as well, but its design came with a whole new set of challenges.
Stockfish, being a C++ application, can't easily be hosted on the internet.
An intermediate REST API was therefore written and containerized, to prepare Stockfish for deployment.
For the final containerized Stockfish application, Google Cloud Run~\cite{google-cloud-run} was decided on as the
hosting provider, since it features easy and relatively cheap deployment of container applications.
