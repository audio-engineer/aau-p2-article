% textidote: ignore begin
\chapter{Conclusion}\label{ch:conclusion}
% textidote: ignore end

The game of chess is one that has stood the test of time.
This is partly due to the highly intricate mechanics that make chess an unsolved game.
As chess is a game where the goal is winning, much effort and time has gone into experimenting with
strategies and new inventions.
The latest step in chess has been computer chess engines which reaches beyond human skill with Deep Blue among others.
Engines are now widely used for analysis.
The authors of this paper have found that they also have immense potential for teaching chess.

Teaching in chess usually comes in the form of a coach.
Even though this approach is one of the most effective teaching tools, it has its drawbacks.
Coaching typically costs money and needs to be planned ahead of time.
This demotivates people from pursuing learning using this teaching method.
Instead, they opt for other cheaper or more accessible solutions.
This gap can be solved using chess engines for learning.

State-of-the-art engine implementation in web apps do not allow online multiplayer integration, which does not
incorporate the social aspects of chess while learning with engines.
In this project the authors have come to the conclusion that this interaction between two chess players is very
important both for the learning experience, but also the enjoyment in learning.
Making the solution into a web application further improves the accessibility and allows people to play on demand.

ChessTeacher does not cater to the completely novice players, as these are players who do not know how or when specific
moves can be made.
The users ChessTeacher caters most to are beginners to intermediates, as they understand the rules and have begun
learning patterns, which is another aspect wherein chess engines are superior to humans.

While finishing the prototype, tests were conducted to understand how feasible the idea behind the project is, and also
how good the implementation is.
Although the reception of ChessTeacher was mixed, most of the critical feedback was related to the project only being
a prototype and not a completely finished and polished product.
This approach has enough merit to be expanded and evolved further in future iterations.

The solution for this project is not perfect.
Important learning aspects of the engine coaching has not been implemented in ChessTeacher.
This is the way the feedback is presented to the players, as currently the users are only presented with the best moves
and no context to why this move is good.
The main limiting factor for this, is that chess engines are not made to communicate context of good moves, but only to
find them.
This means that the engines are not able to explain why it chooses the moves.
This step of the board analysis is just as critical as the actual move the engines suggest for learning.
To solve this a system that outputs human-readable text in context to some input is needed.
One rapidly evolving technology that could potentially cover the shortcomings are Large-Language-Models (LLM's).

Overall ChessTeacher has succeeded in exploring the viability of online multiplayer learning to newer players,
with the possibility of helping more experienced players as well.
This project opens the doors for further exploration of AI powered teaching, building the foundation for a future
where learning chess is accessible, interactive and enjoyable for a thriving chess community.
