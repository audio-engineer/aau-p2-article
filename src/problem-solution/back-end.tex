% textidote: ignore begin
\section{Back end}\label{sec:back-end}
% textidote: ignore end

% textidote: ignore begin
\subsection{Stockfish binary}\label{subsec:stockfish-binary}
% textidote: ignore end

To accompany the chess engine implementation requirement from the software requirement specification seen in
Section~\ref{sec:software-requirement-specification}, Stockfish needs to be implemented.
As Stockfish is a binary program, it is not possible to run it in the browser, why an alternative solution using a
Google Cloud Run container is used.
To use Google Cloud Run an image needs to be created.
Reference is made to our Stockfish image in Appendix A for the repository wherein the source code for the image is
contained.

Another issue with the Stockfish binary is that it provides a lot of information when its built-in functions are
executed.
For our use case, we only need a few specific key data values to later display on the frontend.
One solution to this problem is to run the binary and pipe all the output information directly to the frontend and
filter through it in the frontend.
The advantages of this method is that the developers would be able to keep using TypeScript for the programming of the
data processing as well as the other frontend elements instead of incorporating other languages and data processing
methods.
However, the disadvantages is that this solution uses a lot of bandwidth when communicating the information to the
frontend, which in turn becomes expensive rather quickly.
Instead, the container running the image parse the data locally and only post the information to the frontend that is
requested.

To accomplish this, a Python script has been included in the image containing the Stockfish binary.
The script uses a FastAPI based web API to interact with the Stockfish binary.

To make sure the user is not waiting for an unnecessarily long time for the analysis of the position and to make sure
that there is at least some substance in the depth of the analysis, a hard-coded midpoint of depth 8 is set by the
developers for the Stockfish binary.

The two main functionalities of the script are the functions \texttt{evaluate\_position} and \texttt{analyze\_move}.

\subsection{Stockfish image functions}\label{subsec:stockfish-image-functions}

The position is as before mentioned communicated through a FEN (Forsyth-Edwards Notation) string.
The FEN string is produced by the frontend and is included in the request for the position evaluation.

% textidote: ignore begin
\subsubsection{Function: \texttt{evaluate\_position}}\label{subsubsec:function:evaluate_position}
% textidote: ignore end

This function's main purpose is to evaluate a FEN position.
The FEN position is then evaluated by Stockfish.
A snippet of the function can bee seen in Listing~\ref{lst:main.py-evaluate_position}, where the function definition is
listed.

\begin{lstlisting}[
    style=pythonStyle,
    label={lst:main.py-evaluate_position},
    caption={Definition of function \texttt{evaluate\_position} in \texttt{main.py}},
    captionpos=b,
]
@app.post("/evaluate-position")
async def evaluate_position(
    evaluate_position_request: EvaluatePositionRequest,
) -> DataResponse[EvaluatePositionResponse]:
\end{lstlisting}

In the definition, it can be seen that the function has a single formal parameter \texttt{evaluate\_position\_request},
which is of type \texttt{EvaluatePositionRequest}, which encapsulates a FEN string and a time tracking variable.
It returns an object of type \texttt{EvaluatePositionResponse}, which contains an evaluation of the FEN position,
statistics regarding whether the game was won, drawn or lost, and a list of the top three recommended moves.

% textidote: ignore begin
\subsubsection{Function: \texttt{analyze\_move}}\label{subsubsec:function:analyze_move}
% textidote: ignore end

This function's main purpose is to analyze a single move.
A snippet of the function can bee seen in Listing~\ref{lst:main.py-analyze_move}, where the function definition is
listed.

\begin{lstlisting}[
    style=pythonStyle,
    label={lst:main.py-analyze_move},
    caption={Definition of function \texttt{analyze\_move} in \texttt{main.py}},
    captionpos=b,
]
@app.post("/analyze-move")
async def analyze_move(
    analyze_move_request: AnalyzeMoveRequest,
) -> DataResponse[AnalyzeMoveResponse]:
\end{lstlisting}

As before, in the definition, it can be seen that the function has a single formal parameter
\texttt{analyze\_move\_request}, which is of type \texttt{AnalyzeMoveRequest}, which encapsulates a FEN string and a
time tracking variable.
In addition, it also encapsulates a move string, wherein the move to be analyzed is located.
It returns an object of type \texttt{AnalyzeMoveResponse}, which contains an evaluation of the FEN position after the
move is made, whether the move was a capture or not, and the absolute change in evaluation between the pre- and
post-move FEN positions.
