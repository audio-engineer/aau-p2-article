\subsection{Unit testing}\label{subsec:unit-testing}
When creating the functions for the REST API, unit tests were created to ensure that the functions worked as intended.
The unit tests were created using FASTAPI's built-in testing functionality called \texttt{TestClient}.

\begin{lstlisting}[
    style=pythonStyle,
    label={lst:test_main.py-evaluate_position},
    caption={Test of \texttt{evaluate-position} in \texttt{test\_main.py}},
    captionpos=b,
]
def test_evaluate_position():
    response = client.post(
        "/evaluate-position",
        json={"fen": "rnbqkbnr/pppppppp/8/8/8/8/PPPPPPPP/RNBQKBNR w KQkq - 0 1"},
    )
    assert response.status_code == 200
\end{lstlisting}

In Listing~\ref{lst:test_main.py-evaluate_position}, a test of the \texttt{evaluate-position} function is shown.
The test uses a client object to send a POST request to the \texttt{evaluate-position} endpoint with a FEN string as
the payload.
The test then asserts that the response status code is 200, which means that the request was successful.

The purpose of this test is to first ensure that the \texttt{/evaluate-position} endpoint is reachable and can process
requests.
Secondly, it ensures that the endpoint can handle a FEN string as input and return a response.
Lastly, it ensures that the response status code is 200, which means that the request was successful.

\pagebreak

Below is an example of a test that checks if the endpoint returns a status code of 400 when an invalid FEN
string is provided.

\begin{lstlisting}[
    style=pythonStyle,
    label={lst:test_main.py-evaluate_position-invalid-fen},
    caption={Test of \texttt{evaluate-position} in \texttt{test\_main.py} for invalid FEN string},
    captionpos=b,
]
def test_evaluate_position_invalid_fen():

    response = client.post("/evaluate-position", json={"fen": "fen-position"})

    assert response.status_code == 400
    assert response.json() == {"detail": "Invalid FEN"}
\end{lstlisting}

In Listing~\ref{lst:test_main.py-evaluate_position-invalid-fen}, a test of the \texttt{evaluate-position} function is
shown.
%textidote: ignore begin
The purpose of this test is to ensure the endpoint returns a status code of 400 and a JSON response with the
%textidote: ignore end
%textidote: ignore begin
message \texttt{''Invalid FEN''} when an invalid FEN string is provided.
%textidote: ignore end
This ensures that the endpoint can handle invalid input and return an appropriate error response.
