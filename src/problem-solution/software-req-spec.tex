% textidote: ignore begin
\section{Software requirement specification}\label{sec:software-requirement-specification}
% textidote: ignore end

To answer the problem statement, an outline of the desired program features is made.
The method of choice for the requirement specifications is the MoSCoW model~\cite{hudaib2018}.
The overview of the features the application will and will not entail are provided in Table~\ref{tab:srs}.
In the table, the MoSCoW categories have the following meaning:

\begin{itemize}
    \item \textbf{``Must-haves''} (Mo) encapsulates all the features that must be
    included in the final product.
    \item \textbf{``Should-haves''} (S) encapsulates features that are desired and that the team is technically able to
    implement if not for the time constraints of the project.
    \item \textbf{``Could-haves''} (Co) encapsulates features that are possible
    to implement but are not the focus of this report.
    \item \textbf{``Won't-haves''} (W) encapsulates out of scope features that are not going to be considered nor
    implemented.
\end{itemize}

% textidote: ignore begin
\begin{table}[H]
    \centering
    \resizebox{\columnwidth}{!}{%
        \begin{tabular}{ccc}
            \toprule
            \textbf{Requirement}                     & \textbf{MoSCoW}          & \textbf{Motivation}
            \\ \midrule
            Basic chess functionality                & Mo                       & {\ul~\ref{sec:chess}}
            \\ \midrule
            Online multiplayer functionality         & Mo                       & {\ul~\ref{subsec:multiplayer}}
            \\ \midrule
            Applied learning principles              & Mo                       &
                {\ul~\ref{sec:learning-methods-for-chess}}
            \\ \midrule
            Chess engine implementation              & Mo                       & {\ul~\ref{sec:chess-engines}}
            \\ \midrule
            User accounts                            & S                        & {\ul~\ref{subsec:multiplayer}}
            \\ \midrule
            User registration and authentication     & S                        & {\ul~\ref{subsec:multiplayer}}
            \\ \midrule
            Engine feedback in English               & S                        & {\ul~\ref{sec:chess-engines}}
            \\ \midrule
            Chat functionality                       & S                        &
                {\ul~\ref{subsec:multiplayer}}
            \\ \midrule
            Game listing                             & S                        & {\ul~\ref{subsec:multiplayer}}
            \\ \midrule
            Analysis tool for post game analysis     & Co                       & {\ul~\ref{sec:chess-engines}}
            \\ \midrule
            Chess puzzles                            & Co                       & {\ul~\ref{sec:state-of-the-art}}
            \\ \midrule
            Achievements                             & Co                       & {\ul~\ref{sec:state-of-the-art}}
            \\ \midrule
            Board customization                      & Co                       & {\ul~\ref{sec:state-of-the-art}}
            \\ \midrule
            AI vs AI matches                         & W                        & {\ul~\ref{subsec:ai-engines}}
            \\ \midrule
            Chess variants other than standard chess & W                        & {\ul~\ref{sec:chess}}
            \\ \midrule
            AI training and machine learning         & W                        & {\ul~\ref{subsec:ai-engines}}
            \\ \bottomrule
        \end{tabular}%
    }
    \caption{Software requirements specification in the MoSCoW model.
    The Requirement column describes the individual requirements for the product.
    The MoSCoW column describes the category to which the requirements belong.
    The motivation column describes where in the problem analysis the motivation for the requirements is located.
    }\label{tab:srs}
\end{table}
% textidote: ignore end
