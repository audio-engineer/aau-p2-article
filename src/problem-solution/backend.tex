% textidote: ignore begin
\section{Back end}\label{sec:backend}
% textidote: ignore end

The back end of the application consists of multiple parts: A real-time NoSQL database~\cite{nosql}, authentication
database, Stockfish-powered REST API, and serverless edge functions.

In the following sections, we will take a more in-depth look at some of those components.

% textidote: ignore begin

\subsection{Real-time database}\label{subsec:real-time-database}
% textidote: ignore end

The real-time database is provided through the Google Firebase Realtime Database~\cite{realtime-database} service.
This service provides an easy to integrate ``real-time database'', i.e., a database whose primary job is to synchronize
data between database clients, which, in our case, are the client front ends.
In other words, when data changes on one client, this data can easily be pushed to the remote cloud-hosted database,
which then, in turn, notifies the other clients of the data mutation.
This process ensures data synchronization between application clients.
It is achieved through the database's built-in WebSocket server, which lets the clients create a real-time connection
to the database.
The way that this feature is leveraged in our solution application is that it's used for updating the chessboard, lobby,
and chat states on each of the clients.
The chessboard piece state, without a doubt, has to be instantly synchronized between the players' and spectators'
clients, to create a smooth game experience.
The same goes for the in-game chat, and the lobby, which should show all the created and running matches in real time,
so that signed-in users easily can join or view a match.

% textidote: ignore begin

\subsection{Stockfish API}\label{subsec:stockfish-api}
% textidote: ignore end

To fulfill the chess engine implementation requirement from the software requirement specification seen in
Section~\ref{sec:software-requirement-specification}, an instance of Stockfish needed to be added to the solution.
As Stockfish is binary executable software written in C++, it is not possible to run it in the browser,
unless it's compiled to WebAssembly~\cite{web-assembly}, which can natively run in modern browsers.
The WebAssembly approach was not used in favor of a simpler REST API approach,
which features a simple Python script running in a containerized cloud environment.
This Python script calls a Stockfish wrapper library, which in turn calls Stockfish, and passes the response back to
the Python script, which generates an HTTP response.
The containerized environment to go the with script was created to ease the development and deployment processes.
Containerized applications can easily be built and started for local development, and also easily be built and deployed
to the cloud.
For the cloud hosting of this container image, Google Cloud Run~\cite{google-cloud-run} was selected.
Reference is made to our Stockfish image in Appendix A for the repository wherein the source code for the image is
contained.

Another issue with the Stockfish software is that it provides a lot of information when its built-in functions are
executed.
For our use case, we only require a few specific key data values to later display on the front end.
One solution to this problem is to run the binary and pipe all the output information directly to the frontend and
filter through it in the frontend.
The advantages of this method is that the developers would be able to keep using TypeScript for the programming of the
data processing as well as the other frontend elements instead of incorporating other languages and data processing
methods.
However, the disadvantages is that this solution uses a lot of bandwidth when communicating the information to the
frontend, which in turn becomes expensive rather quickly.
Instead, the container running the image parse the data locally and only post the information to the frontend that is
requested.

To accomplish this, a Python script has been included in the image containing the Stockfish binary.
The script uses a FastAPI based web API to interact with the Stockfish binary.

To make sure the user is not waiting for an unnecessarily long time for the analysis of the position and to make sure
that there is at least some substance in the depth of the analysis, a hard-coded midpoint of depth 8 is set by the
developers for the Stockfish binary.

The two main functionalities of the script are the functions \texttt{evaluate\_position} and \texttt{analyze\_move}.

\subsection{Stockfish image functions}\label{subsec:stockfish-image-functions}

The position is as before mentioned communicated through a FEN (Forsyth-Edwards Notation) string.
The FEN string is produced by the frontend and is included in the request for the position evaluation.

% textidote: ignore begin

\subsubsection{Function: \texttt{evaluate\_position}}\label{subsubsec:function:evaluate_position}
% textidote: ignore end

This function's main purpose is to evaluate a FEN position.
The FEN position is then evaluated by Stockfish.
A snippet of the function can bee seen in Listing~\ref{lst:main.py-evaluate_position}, where the function definition is
listed.

\begin{lstlisting}[
    style=pythonStyle,
    label={lst:main.py-evaluate_position},
    caption={Definition of function \texttt{evaluate\_position} in \texttt{main.py}},
    captionpos=b,
]
@app.post("/evaluate-position")
async def evaluate_position(
    evaluate_position_request: EvaluatePositionRequest,
) -> DataResponse[EvaluatePositionResponse]:
\end{lstlisting}

In the definition, it can be seen that the function has a single formal parameter \texttt{evaluate\_position\_request},
which is of type \texttt{EvaluatePositionRequest}, which encapsulates a FEN string and a time tracking variable.
It returns an object of type \texttt{EvaluatePositionResponse}, which contains an evaluation of the FEN position,
statistics regarding whether the game was won, drawn or lost, and a list of the top three recommended moves.

% textidote: ignore begin

\subsubsection{Function: \texttt{analyze\_move}}\label{subsubsec:function:analyze_move}
% textidote: ignore end

This function's main purpose is to analyze a single move.
A snippet of the function can bee seen in Listing~\ref{lst:main.py-analyze_move}, where the function definition is
listed.

\begin{lstlisting}[
    style=pythonStyle,
    label={lst:main.py-analyze_move},
    caption={Definition of function \texttt{analyze\_move} in \texttt{main.py}},
    captionpos=b,
]
@app.post("/analyze-move")
async def analyze_move(
    analyze_move_request: AnalyzeMoveRequest,
) -> DataResponse[AnalyzeMoveResponse]:
\end{lstlisting}

As before, in the definition, it can be seen that the function has a single formal parameter
\texttt{analyze\_move\_request}, which is of type \texttt{AnalyzeMoveRequest}, which encapsulates a FEN string and a
time tracking variable.
In addition, it also encapsulates a move string, wherein the move to be analyzed is located.
It returns an object of type \texttt{AnalyzeMoveResponse}, which contains an evaluation of the FEN position after the
move is made, whether the move was a capture or not, and the absolute change in evaluation between the pre- and
post-move FEN positions.
