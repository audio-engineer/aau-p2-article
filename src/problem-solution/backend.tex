% textidote: ignore begin
\section{Back end}\label{sec:backend}
% textidote: ignore end

The back end of the application consists of multiple parts: A real-time NoSQL database~\cite{nosql}, authentication
database, Stockfish-powered REST API, and serverless edge-functions.
The following sections will take a more in-depth look at some of those components.

\subsection{Real-time database}\label{subsec:real-time-database}

The real-time database is provided through the Google Firebase Realtime Database~\cite{realtime-database} service.
This service provides an easy to integrate ``real-time database'', i.e., a database whose primary job is to synchronize
data between database clients, which, in our case, are the client front ends.
In other words, when data changes on one client, this data can easily be pushed to the remote cloud-hosted database,
which then, in turn, notifies the other clients of the data mutation.
This process ensures data synchronization between application clients.

It is achieved through the database's built-in WebSocket server, which lets the clients create a real-time connection
to the database.
The way that this feature is leveraged in our solution application is that it's used for updating the chessboard, lobby,
and chat states on each of the clients.
The chessboard piece state has to be instantly synchronized between the players' and spectators'
clients to create a smooth game experience.
The same goes for the in-game chat, and the lobby, which should show all the created and running matches in real time,
so that signed-in users easily can join or view a match.

\subsection{Stockfish API}\label{subsec:stockfish-api}

To fulfill the chess engine implementation requirement from the software requirement specification seen in
Section~\ref{sec:software-requirement-specification}, an instance of a chess engine needs to be added to the solution.
This presents various questions as e.g., what engine to use, and how to connect it with the rest of the solution.
The development team considered multiple options,
but settled on the Stockfish~\cite{stockfish} engine for a variety of reasons, which will be explained in the following
section.

\subsubsection{Rationale}

Stockfish is arguably one of the strongest chess engines~\cite{about-stockfish} on the market right now,
which makes it extremely desirable sheerly due to its processing power.
Stockfish has, since 2013, participated in the Top Chess Engine Championship (abbr.\ TCEC) and ``has finished first or
second in every season except one''~\cite{tcec-results,chess-com-stockfish},
and this result just proves Stockfish's capabilities.

% textidote: ignore begin % Due to `sh:d:002` on `\texttt{Chess.com}` and `\texttt{lichess.org}`
Furthermore, \texttt{Chess.com} and \texttt{lichess.org}, two of our main competitors, also extensively make use of
% textidote: ignore end
Stockfish in their own implementations~\cite{chess-com-stockfish,about-lichess}, which is another remarkable indicator
of Stockfish's capabilities and usage, as mentioned in Section~\ref{sec:state-of-the-art}.

However, arguably one of the most compelling reasons for using Stockfish is the fact that the engine is free and open
source, protected by the GPLv3 license~\cite{gplv3}.
The GPLv3 license is quite permissible, i.e., the user can freely use and even modify the original source code, as long
as the child software refers back to the license declaration and code repository of the licensed parent software.
In the case of the \texttt{ChessTeacher} software, it means that the addition of Stockfish is fully permissible within
the bounds of law.
The source code repository for Stockfish can be found on its official GitHub repository~\cite{github-stockfish}, from
where the source code can easily be downloaded and compiled using any C++-compatible compiler.

These three main reasons compel the software team to pick Stockfish as the driving chess engine for the back end.
The following sections explain more concretely how Stockfish is connected to the rest of the software.

\subsubsection{Usage}

As Stockfish is binary executable software written in C++, it is not possible to run it in the browser,
unless it's compiled to WebAssembly~\cite{web-assembly}, which can natively run in modern browsers.
The WebAssembly approach was dropped in favor of a simpler REST API approach,
which features a simple Python script running in a containerized cloud environment.

As a brief overview; on receiving an HTTP request, the Python script calls a Stockfish wrapper library,
which in turn calls Stockfish and passes the response back to the Python script, which generates an HTTP response.
The reasons for choosing this architecture and data flow are multiple:
One of the core issues with the Stockfish software is that it outputs an overwhelming amount of plain text data to the
standard output stream when its built-in functions are called through its command line interface.
For our use case, the team only requires a few specific key data values from Stockfish to later display it on the
browser front end.
One solution to this problem could have been to execute Stockfish's CLI on the back end server, and stream its output
data directly to the front end through TCP or UDP\@.
Then, the data would have to be parsed and filtered on the front end, to extract the necessary data points.
The advantage of this method is that the developers would be able to keep using TypeScript for the programming of the
data processing as well as the other front end elements.
However, one of the disadvantages of this approach is that it uses a lot of network data when communicating the
information to the front end, which in turn is financially unsustainable.
Instead, the container server running Stockfish and the Python script parses the data locally and only returns the data
to the front end that is needed for the application to function.

The containerized environment to go with the script and Stockfish is created to ease the development and deployment
processes.
Containerized applications can easily be built and started for local development, and also easily be built and deployed
to the cloud.
For the cloud hosting of this container image, Google Cloud Run~\cite{google-cloud-run} was selected, as it features
a free tier for simple container images.
Reference is made to our Stockfish image in Appendix~\ref{ch:source-code-repositories} for the repository wherein the
source code for the image is contained.

To make sure the client is not waiting unnecessarily for the analysis of the position, while also retaining some depth
to the board state analysis, the developers set a hard-coded midpoint of depth 8 for the Stockfish software.

\subsection{Stockfish API functions}\label{subsec:stockfish-api-functions}

The two main functionalities of the script are the functions \texttt{evaluate\_position} and \texttt{analyze\_move}.

The position is communicated through a FEN (Forsyth-Edwards Notation) string.
The FEN string is produced by the frontend and is included in the request for the position evaluation.

% textidote: ignore begin

\subsubsection{Function: \texttt{evaluate\_position}}\label{subsubsec:function:evaluate_position}
% textidote: ignore end

This function's main purpose is to evaluate a FEN position.
The FEN position is then evaluated by Stockfish.
A snippet of the function can be seen in Listing~\ref{lst:main.py-evaluate_position}, where the function definition is
listed.

\begin{lstlisting}[
    style=pythonStyle,
    label={lst:main.py-evaluate_position},
    caption={Definition of function \texttt{evaluate\_position} in \texttt{main.py}},
    captionpos=b,
]
@app.post("/evaluate-position")
async def evaluate_position(
    evaluate_position_request: EvaluatePositionRequest,
) -> DataResponse[EvaluatePositionResponse]:
\end{lstlisting}

In the definition, it can be seen that the function has a single formal parameter \texttt{evaluate\_position\_request},
which is of type \texttt{EvaluatePositionRequest}, which encapsulates a FEN string and a time tracking variable.
It returns an object of type \texttt{EvaluatePositionResponse}, which contains an evaluation of the FEN position,
statistics regarding whether the game was won, drawn or lost, and a list of the top three recommended moves.

% textidote: ignore begin

\subsubsection{Function: \texttt{analyze\_move}}\label{subsubsec:function:analyze_move}
% textidote: ignore end

This function's main purpose is to analyze a single move.
A snippet of the function can be seen in Listing~\ref{lst:main.py-analyze_move}, where the function definition is
listed.

\begin{lstlisting}[
    style=pythonStyle,
    label={lst:main.py-analyze_move},
    caption={Definition of function \texttt{analyze\_move} in \texttt{main.py}},
    captionpos=b,
]
@app.post("/analyze-move")
async def analyze_move(
    analyze_move_request: AnalyzeMoveRequest,
) -> DataResponse[AnalyzeMoveResponse]:
\end{lstlisting}

As before, in the definition, it can be seen that the function has a single formal parameter
\texttt{analyze\_move\_request}, which is of type \texttt{AnalyzeMoveRequest}, which encapsulates a FEN string and a
time tracking variable.
In addition to that, it also encapsulates a move string, wherein the move to be analyzed is located.
It returns an object of type \texttt{AnalyzeMoveResponse}.
This object contains an evaluation of the board state after the move is made, whether the move was a capture or not,
and the absolute change in evaluation between the pre- and post-move board positions.
