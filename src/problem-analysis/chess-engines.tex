% textidote: ignore begin
\section{Chess engines}\label{sec:chess-engines}
% textidote: ignore end

Chess engines are defined by~\url{Chess.com} as such:

\begin{quote}
    A chess engine is simply a software program that plays and analyzes chess.
    The word “engine” simply refers to a kind of high-powered program that does a
    lot of searching and processing similar to a search engine~\cite{chess-com-chess-engines}.
\end{quote}

The goal of a chess engine is to understand all the game mechanics and then evaluate a chess position to output some
information about said position.
This evaluation is the most important thing a chess engine should be able to do.
As this is a very abstract problem to solve, there exist many ways to achieve solutions which will be discussed in
the following sections.

\subsection{Engines and their importance for chess web apps}\label{subsec:webapps-engines}

Chess web applications rely on engines to provide many of its core features.
Engines provide the core part of features such as gameplay analysis,
move recommendation and opponent simulation.
This is why it is relevant to understand how these engines work,
to get a clearer picture of how they are used in chess web applications,
and what benefits they bring in context to user engagement and skill
development.
The most important feature engines enable, is the chessboard analysis.
Users can use the analysis tool to understand when and how they make mistakes,
this can help them improve all aspects of their play style.
Engines make use of methods described below, to evaluate a chess position.

Additionally, engines make it much easier to implement learning modules
and mini-games into the web app.
More advanced engines also have the capability to change skill level,
this can be used for training as the individual can choose the difficulty of
their opponent.
To summarize, engines are critical for modern chess web apps as they
expand how the web app can interact with the users.
This interaction will both increase usability, but also makes guidance accessible
24/7 for users.



\subsection{Conventional/brute force engines}\label{subsec:conventional-engines}

A conventional chess engine can be simply described as a brute force approach to solve chess.
This means that the main limiting factor of a chess engine's output quality is the computational power of the system in
which it is executed.
Such a chess engine consists of two parts, an evaluation function and a search function.
Chess is a very complex game without a clear-cut solution to most positions.
Brute forcing all the possible legal moves of a position and moves after those moves, and so on, would require searching
up to~\( 2 \times 10^{42} \) different positions~\cite{steinerberger2014}.
With current technology, this would be impossible to compute.

To ease this computational burden, chess engines have a value called depth or search depth.
The depth value can be understood as how deep the engine should look into the future.
Going to the end of a move chain is not practical, as mentioned before, to give the best move at some current position.

The search function can be further optimized in different ways.
However, optimizations are, in essence, clever ways to search for moves that are more likely to be good before others.
It will also allow cutting out moves that are guaranteed to be worse than the already searched move.

Evaluation of a chessboard can also be done in a lot of different ways, but that is only relevant in terms of
implementation.
What is important to understand for the evaluation is how it is communicated.
To represent the state of the game in any position, a single number is often used, which
is in the range of some arbitrary rational numbers~\( -a : b\).
This is optimal because it allows a single number to represent the score of two people, where~\(-a\) would be
the best score for black, and~\(b\) would be the best score for white.
This representation is possible because chess is a zero-sum game when only playing the best moves, which means that
there is at least one optimal way to play for both white and black at any given position~\cite{zero-sum-games}.

% TODO Maybe add some example like explaining mini-max, though that could be argued to be in /method/problem solution.

A conventional engine would, given some position, search through all relevant moves to some set depth, and then provide
the best move, or a list of ranked moves based on the evaluation of the positions.

\subsection{AI engines}\label{subsec:ai-engines}

Chess engines that implement neural networks to both search and/or evaluate, called AI engines, have been making the
news in many kinds of board games.
This reason for this attention is that AI engines have not only beaten top conventional engines,
but also created new strategies and creative moves that do well against previously good strategies.

This creativity comes from how they are built.
More specifically, how the neural networks are trained, because as opposed to humans who learn by playing against other
humans that use already conventional strategies, AI engines learn by playing against itself many times.

This is the key difference between conventional engines and AI engines.
Conventional engines are based on the creator's knowledge of the game and the programmer's strategies.
AI engines are self-taught, which means that although it will make a lot of errors in strategy at first, the engine will
not be held back by faulty paradigms, which could hamper the potential of its capabilities.
This method is also called reinforcement learning~\cite{how-neural-network-engines-work}.

Another difference to conventional engines is that AI engines do not measure the evaluation the same way.
The rating is instead based on the probability of winning.
This does not change how it chooses moves as it will still pick the best of all the futures it has searched.

\subsection{Stockfish}\label{subsec:stockfish}

Conventional engines began to get their renown as beyond human capabilities, when the then chess world champion
Garry Kasparov had multiple matches against IBM's chess engine Deep Blue in both 1996 and
1997~\cite{Garry-Kasparov-vs-Deep-Blue}.
The matches played in 1996 ended in a 4--2 win for Kasparov, which was already a shock to the world as the man vs.\
machine battle became a more serious topic.
The man vs.\ machine battle came to a head the next year, when Kasparov lost 3½--2½ against Deep Blue, which meant that
man was no longer the best.
This trend has only continued, and a much more modern chess engine, Stockfish, has emerged after 16 years of development
since its initial release in 2008.
Stockfish started as a conventional chess engine, but has since 2020 included neural network
evaluation (NNUE Evaluation) in the source code.
The introduction of neural networks has brought Stockfish's ELO up in the 3600 range~\cite{Stockfish}.
This can be compared to the current highest rated chess player Magnus Carlsen, who has a
rating of 2882~\cite{highest-ELO-chess-player}.

Stockfish is also an open source project which means that everyone has access to it.
The engine is also well documented which makes Stockfish a great choice for solving the problems mentioned earlier.

% textidote: ignore begin
\subsection{AlphaZero}\label{subsec:alphazero}
% textidote: ignore end

The most relevant example of an AI chess engine comes from Google Deepmind's Alpha Zero.
Alpha Zero was an AI first introduced in the form of AlphaGo Zero in 2016.
This was the first version of the AI shown to the world.
It was an AI made to play the ancient board game Go.

Go is a board game played on a nine-by-nine grid, and is arguably a more challenging game than chess to make an engine
around, because of the many possible positions and moves.
It was thought impossible for computers to surpass humans until AlphaGo's introduction.

AlphaGo stunned the Go community when it played a best-of-five match against the professional
Go player Lee Sedol, who was the second-highest ranked Go player in the world.
AlphaGo won 4--1 against Lee Sedol~\cite{alpha-go-zero}, but AlphaGo was not only made to play Go, another more powerful
version was in the works.

As mentioned in Section~\ref{subsec:stockfish}, Stockfish was before 2017 thought to be the strongest chess engine,
and as of 2024 now is again.
However, when AlphaZero played against the Stockfish engine in 2017, although shocking for the chess community,
AlphaZero beat Stockfish with a convincing lead.
This result showed that conventional engines were not the end of the road for playing chess.
AlphaZero's success has since then inspired engines like Stockfish and Leela Chess Zero to build and improve on the
ideas introduced with AlphaZero.
