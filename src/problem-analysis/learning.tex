% textidote: ignore begin
\section{Learning methods for chess}\label{sec:learning-methods-for-chess}
% textidote: ignore end

When learning, a lot of different methodologies and techniques can be applied.
Mentioned in the following sections are the core elements for learning chess.

% textidote: ignore begin

\subsection{Trial and error}\label{subsec:trial-and-error}
% textidote: ignore end

One learning method is the trial and error methodology.
For this way of learning to be effective, the subject, i.e., the student, needs to be able to make mistakes and receive
instant feedback about either the consequences or the benefits of the action~\cite{li2023}.
In the case of chess, the student must be able to make mistakes.
These mistakes can range from short-term mistakes, blunders, to long-term mistakes such as positional mistakes.
In addition, the rewards for performing good moves need to encapsulate the reason for it being a good move, such that
the student understands the benefits and can compare their thought process to that of the feedback.
However, this method alone is, in our opinion, not enough to facilitate effective learning.
Instead, assisted learning can be applied.

% textidote: ignore begin

\subsection{Assisted learning}\label{subsec:assisted-learning}
% textidote: ignore end

To learn, someone or something needs to teach.
As mentioned, one can learn solely from their environment with the trial and error methodology.
However, this procedure can involve unnecessary insignificant learning moments that instead potentially could have been
optimized by someone or something that has already experienced the lessons that were taught.
It can therefore be in the interest of a student to be taught by someone with more experience and expertise than them,
by way of a more authoritarian approach.
A combination of these methodologies can have a greater impact on the learning of a student than using them
individually~\cite{swann2012}.
The advantages of a physical chess teacher are among others that they potentially have experience in both the material
they are teaching and in teaching methodologies, thereby streamlining and focusing the learning process for the student.
However, there are some drawbacks with a physical human teacher.
These include, but are not limited to, human fatigue, human emotions, potential personal distractions and so on,
as most all humans experience.
These factors potentially hinder physical teachers from teaching as effectively as is possible.

Studies suggest that peer-assisted learning is more effective than individual learning at higher task complexity and
difficulty~\cite{carson2023}.
Chess is arguably a complex game.
One argument for its complexity is that chess remains an `unsolved' game for the most part.
For a game to be solved, all known best moves for all possible positions of the game need to be identified or be
identifiable relatively easily through algorithms or the like~\cite{herik2002}.
An example of a solved game is checkers~\cite{schaeffer2007}.
Chess with six or fewer pieces left on the board including the kings is solved and encapsulates roughly
only~\( 4 \times 10^{12} \) positions out of all possible positions~\cite{syzygy2024}.
However, chess has an upper bound of~\( 2 \times 10^{40} \) different legal positions, not counting positions where one
or more pawns has promoted~\cite{steinerberger2014}.
This extraordinary large amount of possible positions makes for a greatly challenging game to learn.
It can hereby be observed that peer-assisted learning will likely have a greater impact on individuals learning chess
than individuals learning chess by themselves.

Another factor that improves the effectiveness of learning is the ability to interact with educators and to ask
follow-up questions about a desired topic.
Hereby, it is also important to have a good relationship with one's educators, where both parties feel comfortable with
each other and where a comfortable learning environment is constructed~\cite{saha2009}.
In the context of an online non-person educator, the program should be able to adapt and relate to the student to
produce a more effective learning environment for the student.

To be able to efficiently and effectively learn, one should not only remember but also understand what they are taught.
Explanations are vital for this understanding and is necessary for students to learn effectively~\cite{williams2010}.
In the context of the ChessTeacher project, the program must be able to provide detailed case specific explanations to
the user to effectively facilitate learning.

\subsection{Multiplayer}\label{subsec:multiplayer}

Chess is famously played by two players, where each player takes turns to move their pieces.
Because of this, the game is inherently a multiplayer game.
There are, however, different types of multiplayer games, and while chess is originally a tabletop game, adapting it to
a digital platform affects the multiplayer aspect of the game.
This section will go over the different types of multiplayer games, and how the transition from local to online
multiplayer can have an effect on learning chess.

\subsubsection{Types of multiplayer games}

Multiplayer is a term used to describe games that can be played by more than one player.
It is mostly associated with video games, but most tabletop games, such as chess, are also multiplayer games.
Throughout the years, there have been three dominating types of multiplayer implementations~\cite{multiplayer-types},
and chess can fall into two of these types.

\begin{itemize}

    \item \textbf{Local multiplayer} has the players take turns to play the game.
    It requires the players to be in the same room and to wait for their turn to play.
    Traditional tabletop games, such as chess, fall into this category.

    \item \textbf{Co-op multiplayer} has two players playing on a split-screen.
    It also requires the players to be in the same room, but unlike the first type, they can play at the same time.
    This type is mostly associated with video games, and it is not relevant for chess.

    \item \textbf{Online multiplayer} allows for players to play together over the internet, negating the requirement to
    be in the same room as with the previous two types.
    This type of multiplayer can be seen in online chess games.

\end{itemize}

It can be concluded that by adapting chess to a digital platform, its multiplayer changes from local to online
multiplayer.
While the game itself remains the same, the way players interact with each other changes drastically.
This can have an impact on the experience players have when learning and playing chess.

\subsubsection{Effect multiplayer has on learning}

Online multiplayer has a lot of benefits over local multiplayer.
It makes it possible to connect with people from all over the world, as they don't have to be in the same room.
It also allows for people to play against strangers, usually in a competitive environment, fitting for chess.
However, it takes away from the social aspect of the game, as players rarely communicate with each other when playing
and learning chess online.

There are different arguments on the matter, one source praises how online multiplayer allows for people to connect and
socialize with people they never could before~\cite{multiplayer-online}.
Another source is about how transitioning from local to online multiplayer has significantly impacted the social
interaction with their friend~\cite{multiplayer-local}.
Online multiplayer games often attempt to mitigate the lack of social interaction by implementing chat, voice or video
systems, but it is rare for players, especially strangers, to use these features.

This lack of communication can have a negative effect on learning chess, as it discourages players from reaching out
for help or advice.
That typically results in a poor experience for novices.
A lot of people would rather learn chess from friends, family or a coach, resulting in a local multiplayer environment.
Therefore, we can conclude that the transition from local to online multiplayer has a negative effect on the learning
experience of chess.
