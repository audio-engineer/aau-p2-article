% textidote: ignore begin
\section{Learning methods for chess}\label{sec:learning-methods-for-chess}
% textidote: ignore end

When learning, a lot of different methodologies and techniques can be applied.
Mentioned is the core elements of our learning scheme to be imposed on the subject.

% textidote: ignore begin
\subsection{Trial \& error}\label{subsec:trial-and-error}
% textidote: ignore end

One learning method is the trial and error methodology.
For this way of learning to be effective the subject, i.e.\ the student, needs to be able to make mistakes and receive
instant feedback about either the consequences or the benefits of the action~\cite{li2023}.
In the case of chess, the student must be able to make mistakes.
These mistakes can range from short term mistakes, blunders, to long term mistakes such as positional mistakes.
In addition, the rewards for performing good moves needs to encapsulate the reason for it being a good move, such that
the student understand the benefits and can compare their thought process to that of the feedback.

% textidote: ignore begin
\subsection{Assisted learning}\label{subsec:assisted-learning}
% textidote: ignore end

Studies suggest that peer-assisted learning is more effective than individual learning at higher task complexity and
difficulty~\cite{carson2023}.
Chess is arguably a complex game.
One argument for its complexity is that chess remains an `unsolved' game for the most part.
For a game to be solved all known best moves for all possible positions of the game needs to be identified or be
identifiable relatively easily through algorithms or the like~\cite{herik2002}.
An example of a solved game is checkers~\cite{schaeffer2007}.
Chess with six or fewer pieces left on the board including the kings is solved and encapsulates roughly only
\( 4 \times 10^{12} \) positions out of all possible positions~\cite{syzygy2024}.
However, chess has an upper bound of \( 2 \times 10^{40} \) different legal positions, not counting positions where one
or more pawns has promoted~\cite{steinerberger2014}.
This extraordinary large amount of possible positions makes for a greatly challenging game to learn.
It can hereby be observed that peer-assisted learning will likely have a greater impact on individuals learning chess
than individuals learning chess by themselves.

Another factor that improves the effectiveness of learning is the ability to interact with educators and to ask
follow-up questions about a desired topic.
Hereby it is also important to have a good relationship with one's educators, where both parties feel comfortable with
each other and where a comfortable learning environment is constructed~\cite{saha2009}.
In the context of an online non-person educator, the program should be able to adapt and relate to the student to
produce a more effective learning environment for the student.

To be able to efficiently and effectively learn, one should not only remember but also understand what they are taught.
Explanations are vital for this understanding and is necessary for students to learn effectively~\cite{williams2010}.
In the context of the ChessTeacher project, the program must be able to provide detailed case specific explanations to
the user.
