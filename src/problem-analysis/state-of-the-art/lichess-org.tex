% textidote: ignore begin
\subsection{Lichess.org}\label{subsec:lichess-org}
% textidote: ignore end

\url{Lichess.org} is a free, open-source online chess platform that offers a variety of services to its users.
Lichess was founded in 2010 by Thibault Duplessis, and has since then grown to be one of the most popular chess websites
in the world~\cite{about-lichess}.

Unlike~\url{Chess.com},~\url{Lichess.org} is completely free and open-source meaning that the website is free to use,
and the source
code is available for anyone to use and modify~\cite{Lichess-Github}.
This makes~\url{Lichess.org} a unique platform in the chess community.
Unlike a lot of other chess websites, they do not charge users for their services, while also being an open-source
platform where the community can contribute to the development of the website.

\subsubsection{What does Lichess offer?}\label{subsubsec:what-does-lichess-offer}

As~\url{chess.com},~\url{Lichess.org} offers a wide range of services to its users, such as playing chess,
learning chess, and coaching~\cite{about-lichess}.
On~\url{Lichess.org} you can train your skills by playing against a computer or by simply solving puzzles.
\url{Lichess.org} is also great for learning chess, as it offers almost the same features as \url{chess.com}
such as lessons, puzzles and articles and more.

\url{Lichess.org}, as~\url{Chess.com}, has a way of analyzing games.
Both~\url{Chess.com} and~\url{Lichess.org} use Stockfish as their analysis engine, which makes players able to
refine their skills.
Both Lichess and~\url{Chess.com} also offer a mobile application for their services, which is a great feature for
players who are on the go.

It also has a section for studying new tactics and openings called “Study” where you can create your own studies for
a new opening or strategy~\cite{Lichess.org}.
Lichess focuses more on the community aspect of chess.
As said before, it is a platform for chess players of all skills to share information about chess strategies,
tactics and more.
%textidote: ignore begin
The Lichess community is very active and has a large number of players who are active in forums and chat rooms.
%textidote: ignore end
They host regular tournaments for players to compete in, and like~\url{chess.com} the skill level can vary
from casual to championship level tournaments.
