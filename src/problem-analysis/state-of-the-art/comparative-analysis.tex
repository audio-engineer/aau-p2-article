% textidote: ignore begin
\subsection{Comparative analysis}\label{subsec:comparative-analysis}
% textidote: ignore end

To get a better understanding of the differences and similarities of the state of the art, reference is made to Table
~\ref{tab:comparative-analysis}.
In the table, the explanation of the services are as follows:

\begin{itemize}
    \item \textbf{\url{Chess.com}}: The chess website as explained in Section~\ref{subsec:chess-com}.
    \item \textbf{\url{Lichess.com}}: The chess website as explained in Section~\ref{subsec:lichess-org}.
    \item \textbf{OTB chess}: Short for `over the board chess'.
    The sport of physical chess, where one is accompanied by a psychical chess board.
    \item \textbf{Chess Educator}: The service where an actual physical person is accompanying the user, either online
    or offline, and giving the user feedback and helping vocally by communicating their thoughts about the users moves
    and the game to the user.
    \item \textbf{Our solution}: Encapsulates the solution we produce with the functionalities described in
    Section~\ref{sec:software-requirement-specification}.
\end{itemize}

Also in the table, the explanation of the features are as follows:

\begin{itemize}
    \item \textbf{Multiplayer}: The ability to play with another person either online or offline.
    \item \textbf{Learning}: The ability to achieve strategic knowledge about the game of chess through various learning
    methods.
    \item \textbf{Multiplayer learning}: A combination of the two beforehand mentioned features, with the caveat that it
    can be accomplished simultaneously.
    \item \textbf{Free to use}: The ability to use the service without it costing the user any money.
\end{itemize}

In the table it can be observed that OTB chess entails the possibility of the players learning from each other and play
multiplayer chess when offline, hereby potentially making it the best way to learn chess offline, whilst being able to
play multiplayer and having interactive learning simultaneously.
It can also be observed that our solution will not be able to encompass offline users, as it will be an online web
application.
However, in the online section it is observed that only chess educators and our solution provides the multiplayer
learning experience to the user.
The downside of knowledgeable educators is that they charge money for their teaching service, why our solution will be
the only solution that can potentially satisfy all the features in the table.
In the table is not included all the other services that \url{Lichess.org} and \url{Chess.com} provide, as that is not
the focus of the report.

% textidote: ignore begin
\begin{table}[h]
    \centering
    \resizebox{\columnwidth}{!}{%
        \begin{tabular}{clllll}
            \toprule
            \rowcolor[HTML]{9B9B9B}
            \textbf{Features}                                                        & \multicolumn{5}{c}{\cellcolor[HTML]{9B9B9B}\textbf{Services}}                                                                                                                                                                                                                                                                                    \\ \midrule
            \rowcolor[HTML]{EFEFEF}
            \cellcolor[HTML]{C0C0C0}{\color[HTML]{000000} \textit{\textbf{Online}}}  & \multicolumn{1}{c}{\cellcolor[HTML]{EFEFEF}\textbf{Chess.com}} & \multicolumn{1}{c}{\cellcolor[HTML]{EFEFEF}\textbf{Lichess.com}} & \multicolumn{1}{c}{\cellcolor[HTML]{EFEFEF}\textbf{OTB chess}} & \multicolumn{1}{c}{\cellcolor[HTML]{EFEFEF}\textbf{Chess Educator}} & \multicolumn{1}{c}{\cellcolor[HTML]{EFEFEF}\textbf{Our solution}} \\ \midrule
            \rowcolor[HTML]{67FD9A}
            \cellcolor[HTML]{EFEFEF}\textbf{Multiplayer}                             & \multicolumn{1}{l}{\cellcolor[HTML]{67FD9A}}                   & \multicolumn{1}{l}{\cellcolor[HTML]{67FD9A}}                     & \multicolumn{1}{l}{\cellcolor[HTML]{FD6864}}                   & \multicolumn{1}{l}{\cellcolor[HTML]{67FD9A}}                        &                                                                    \\ \midrule
            \rowcolor[HTML]{67FD9A}
            \cellcolor[HTML]{EFEFEF}\textbf{Learning}                                & \multicolumn{1}{l}{\cellcolor[HTML]{67FD9A}}                   & \multicolumn{1}{l}{\cellcolor[HTML]{67FD9A}}                     & \multicolumn{1}{l}{\cellcolor[HTML]{FD6864}}                   & \multicolumn{1}{l}{\cellcolor[HTML]{67FD9A}}                        &                                                                    \\ \midrule
            \rowcolor[HTML]{FD6864}
            \cellcolor[HTML]{EFEFEF}\textbf{Multiplayer learning}                    & \multicolumn{1}{l}{\cellcolor[HTML]{FD6864}}                   & \multicolumn{1}{l}{\cellcolor[HTML]{FD6864}}                     & \multicolumn{1}{l}{\cellcolor[HTML]{FD6864}}                   & \multicolumn{1}{l}{\cellcolor[HTML]{67FD9A}}                        & \cellcolor[HTML]{67FD9A}                                           \\ \midrule
            \rowcolor[HTML]{67FD9A}
            \cellcolor[HTML]{EFEFEF}\textbf{Free to use}                           & \multicolumn{1}{l}{\cellcolor[HTML]{67FD9A}}                   & \multicolumn{1}{l}{\cellcolor[HTML]{67FD9A}}                     & \multicolumn{1}{l}{\cellcolor[HTML]{67FD9A}}                   & \multicolumn{1}{l}{\cellcolor[HTML]{FD6864}}                        &                                                                    \\ \midrule
            \rowcolor[HTML]{EFEFEF}
            \cellcolor[HTML]{C0C0C0}{\color[HTML]{333333} \textit{\textbf{Offline}}} & \multicolumn{1}{c}{\cellcolor[HTML]{EFEFEF}\textbf{Chess.com}} & \multicolumn{1}{c}{\cellcolor[HTML]{EFEFEF}\textbf{Lichess.com}} & \multicolumn{1}{c}{\cellcolor[HTML]{EFEFEF}\textbf{OTB chess}} & \multicolumn{1}{c}{\cellcolor[HTML]{EFEFEF}\textbf{Chess Educator}} & \multicolumn{1}{c}{\cellcolor[HTML]{EFEFEF}\textbf{Our solution}} \\ \midrule
            \rowcolor[HTML]{FD6864}
            \cellcolor[HTML]{EFEFEF}\textbf{Multiplayer}                             & \multicolumn{1}{l}{\cellcolor[HTML]{67FD9A}}                   & \multicolumn{1}{l}{\cellcolor[HTML]{67FD9A}}                     & \multicolumn{1}{l}{\cellcolor[HTML]{67FD9A}}                   & \multicolumn{1}{l}{\cellcolor[HTML]{67FD9A}}                        &                                                                    \\ \midrule
            \rowcolor[HTML]{67FD9A}
            \cellcolor[HTML]{EFEFEF}\textbf{Learning}                                & \multicolumn{1}{l}{\cellcolor[HTML]{67FD9A}}                   & \multicolumn{1}{l}{\cellcolor[HTML]{67FD9A}}                     & \multicolumn{1}{l}{\cellcolor[HTML]{67FD9A}}                   & \multicolumn{1}{l}{\cellcolor[HTML]{67FD9A}}                        & \cellcolor[HTML]{FD6864}                                           \\ \midrule
            \rowcolor[HTML]{FD6864}
            \cellcolor[HTML]{EFEFEF}\textbf{Multiplayer learning}                    & \multicolumn{1}{l}{\cellcolor[HTML]{FD6864}}                   & \multicolumn{1}{l}{\cellcolor[HTML]{FD6864}}                     & \multicolumn{1}{l}{\cellcolor[HTML]{67FD9A}}                   & \multicolumn{1}{l}{\cellcolor[HTML]{67FD9A}}                        &                                                                    \\ \midrule
            \rowcolor[HTML]{67FD9A}
            \cellcolor[HTML]{EFEFEF}\textbf{Free to use}                           & \multicolumn{1}{l}{\cellcolor[HTML]{67FD9A}}                   & \multicolumn{1}{l}{\cellcolor[HTML]{67FD9A}}                     & \multicolumn{1}{l}{\cellcolor[HTML]{67FD9A}}                   & \multicolumn{1}{l}{\cellcolor[HTML]{FD6864}}                        &                                                                    \\ \bottomrule
        \end{tabular}%
    }
    \caption{Comparative analysis of the services and some features they have.
    A green color indicates that the service encapsulates the feature.}\label{tab:comparative-analysis}
\end{table}
% textidote: ignore end
