\subsection{Engines and their importance for chess web apps}\label{subsec:webapps-engines}

Chess web applications rely on engines to provide many of its core features.
Engines provide the core part of features such as gameplay analysis,
move recommendation and opponent simulation.
This is why it is relevant to understand how these engines work,
to get a clearer picture of how they are used in chess web applications,
and what benefits they bring in context to user engagement and skill
development.
The most important feature engines enable, is the chessboard analysis.
Users can use the analysis tool to understand when and how they make mistakes,
this can help them improve all aspects of their play style.
Engines make use of methods described in Section~\ref{subsec:chess-engines}
to evaluate a chess position.

Additionally, engines make it much easier to implement learning modules
and mini-games into the web app.
More advanced engines also have the capability to change skill level,
this can be used for training as the individual can choose the difficulty of
their opponent.
To summarize, engines are critical for modern chess web apps as they
expand how the web app can interact with the users.
This interaction will both increase usability, but also makes guidance accessible
24/7 for users.

