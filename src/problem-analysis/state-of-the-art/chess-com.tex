% textidote: ignore begin
\subsection{Chess.com}\label{subsec:chess-com}
% textidote: ignore end

\url{Chess.com} is a leading website in the chess community, offering a wide range of services to its users.
The website was founded in 2007, but the idea started in 2005 by two friends~\cite{Chess.com}, and has since then grown
to be one of the most popular chess websites in the world.
Chess.com has many features other than just playing chess, such as lessons, puzzles, and articles.
Chess.com serves as a virtual platform for chess players of all skill levels to unite and play chess against each other.

% textidote: ignore begin
\subsubsection{What does Chess.com offer?}\label{subsubsec:what-does-chess-com-offer}
% textidote: ignore end

\begin{itemize}
    \item \textbf{Play chess}: \url{Chess.com} offers a platform for players to engage in real-time chess games against
    opponents from all over the globe.
    Users can play against friends, random opponents, or the computer at various difficulty levels.
    \item \textbf{Learning Chess}: \url{Chess.com} offers a wide range of lessons and videos to help players
    improve their skills in chess.
    Not only lessons but also puzzles and articles to help players understand the game better and gain more knowledge.
    There are also videos covering various aspects of the game such as openings, endgames and tactics.
    They also offer a feature called Chess coach which is an interactive learning tool that helps players improve
    their game with personalized lessons and exercises.
    \item \textbf{Community}: \url{Chess.com} has a large community of players who are active in forums and chat rooms.
    You would be able to discuss strategies, ask for advice and share your thoughts with other players.
    \item \textbf{Competing}: For those who are more competitive, \url{Chess.com} offers tournaments and leaderboards.
    \url{Chess.com} hosts regular tournaments for players to compete in the tournaments can range from
    casual to championship level tournaments.
    \item \textbf{Analyze}: \url{Chess.com} offers a very powerful tool for analyzing games.
    By analyzing games, players can review their games and learn from their mistakes and make
    better moves in the future.
    This tool is one of the most powerful tools on the website and is used by many players to improve their game.
\end{itemize}
