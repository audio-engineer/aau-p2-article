% textidote: ignore begin
\subsection{Chess engines}\label{subsec:chess-engines}
% textidote: ignore end

Chess engines are defined by \url{Chess.com} as such:

\begin{quote}
    A chess engine is simply a software program that plays and analyzes chess.
    The word “engine” simply refers to a kind of high-powered program that does a
    lot of searching and processing similar to a search engine~\cite{chess-com-chess-engines}.
\end{quote}

So the goal of a chess engine is to understand all the game mechanics and then evaluate a position.
This evaluation is the most important thing a chess engine should be able to do, but as this is a very abstract
problem there exists many ways to solve the problem which will be discussed below.

\subsubsection{Conventional/brute force engines}\label{subsubsec:conventional-engines}

A conventional chess engine can be simply described as a brute force approach to solve chess.
This means that the main limiting factor of how good the engine will be is the computational power of the system.
Such a chess engine consists of two parts, an evaluation function and a search function.
Chess is a very complex game without a clear-cut solution, this is because if you wanted to brute force all the possible
moves that the board could be in you would have to search \( 10^{111} \) different positions, that is more atoms than
there are in the universe, and with current technology this would be impossible to compute~\cite{atoms}.

To ease this computational burden, chess engines have a value called depth or search depth.
The depth value can be understood as how deep the engine should look into the future, as going to the end is not
possible, to give the best move at some current position.

The search function can be further optimized in different ways, but what the optimizations in essence are, are
clever ways to search for moves that are more likely to be good before others, that will allow you to cut out moves that
are guaranteed to be worse than the already searched move.
This can be visualized through a tree structure as seen in (Some figure).

%TODO input the correct figur.

Evaluation of a chessboard can also be done in a lot of different ways, but that is only relevant in terms of
implementation.
What is important to understand for the evaluation is how it is shown.
In normal chess you always have two players, and to represent the state of the game you would use a single number,
that is in the range of some arbitrary numbers \( -a : b\).
This is optimal because it allows a single number to represent the score of two people, where \(-a\) would be
the best score for black, and \(b\) would be the best score for white.
This representation is possible because chess is a zero-sum game with perfect information, which means that there is
at least one optimal way to play for both white and black at any given position~\cite{zero-sum-games}.

% TODO Maybe add some example like explaining mini-max though that could be argued to be in /method/problem solution.

A conventional engine would, given some position then search through all relevant moves to some depth, and then give
the best move, or a list of ranked moves based on the evaluation of the positions.

\subsubsection{AI engines}\label{subsubsec:ai-engines}

Chess engines that implement neural networks to both search and/or evaluate, which we will call AI engines,
have been making the news in many kinds of board games.
This attention comes because that they have not only beaten top conventional engines,
but that they also show new strategies and creative moves that go against previous strategy.

This creativity comes from how they are built, more specifically how the neural networks are trained, because as opposed
to humans who learn by playing against other humans that use already conventional strategies, AI engines learn by
playing against itself many times.

\textbf{This} is the key difference between conventional engines and AI engines, simply put, conventional
engines are based on the creators' knowledge of the game and the programmers strategies.
AI engines are self-taught, which means that although it will have to make a lot of errors in strategy at first.
The engine will not be held back by faulty paradigms, which could hamper the potential of its
capabilities, this method is also called reinforcement learning~\cite{how-neural-network-engines-work}.

Another difference between conventional engines is that AI engines do not measure the evaluation is not numerical, the
rating is instead based on the probability of winning, this does not change how it chooses moves as it will still pick
the best of all the futures it has searched.

\subsubsection{Stockfish}\label{subsubsec:stockfish}

Conventional engines began to get their renown as beyond human capabilities, when the then chess world champion
Garry Kasparov had multiple matches against IBM's chess engine Deep Blue in both 1996 and
1997~\cite{Garry-Kasparov-vs-Deep-Blue}.
The matches played in 1996 ended in a 4--2 win for Kasparov, which was already a shock to the world as the man vs
machine battle became a more serious topic.
The man vs machine battle came to a head the next year, when Kasparov lost 3½--2½ against Deep Blue, which meant that
man was no longer the best.
This trend has only continued and a much more modern chess engine Stockfish is the result of 16 years of development
after its initial release in 2008.
Stockfish started of as a conventional chess engine, but has since 2020 included neural network
evaluation (NNUE Evaluation) in the source code.
The introduction of neural networks has brought Stockfish's ELO up in the 3600 range~\cite{Stockfish}.
This can be compared to the current highest rated chess player Magnus Carlsen, who has a
rating of 2882~\cite{highest-ELO-chess-player}.

Stockfish is also an open source project which means that everyone has access to it, the engine is also well documented
which makes Stockfish an obvious choice if one wanted to implement a well build, if not the best chess engine into a
project.

% textidote: ignore begin
\subsubsection{AlphaZero}\label{subsubsec:alphazero}
% textidote: ignore end

A groundbreaking example of an AI chess engine would be Deepmind's AlphaZero.
AlphaZero shocked the world when another variant beat the then 2nd best Go player Lee Sedol in 2015.
Up until that point Go was thought to be too complicated and the requirement of creative thought seemed too difficult
to obtain with the lack of human intelligence, but this was proven wrong when AlphaZero's predecessor, AlphaGo,
won 4--1 against Lee Sedol.
AlphaZero Continued its fight and beat the Go World Champion.
This cemented AlphaZero's place above human minds~\cite{alpha-go-zero} in context to the three well known board games:
Shogi, Chess and Go.

AlphaZero also fought in chess against the best chess engine in 2017 which was Stockfish 8.
% TODO This section needs some more work.
