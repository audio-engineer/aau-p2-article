\subsection{Engines and their importance for chess web apps}\label{subsec:webapps-engines}

Chess web applications rely on engines to provide many of its core features.
Engines provide the core part of features such as gameplay analysis,
move recommendations and opponent simulation.
It is therefore relevant to understand how these engines work,
to get a clearer picture of how they are used in chess web applications,
and what benefits they have in the context of user engagement and skill
development.
The most important feature engines enable is the chessboard analysis.
Users can use the analysis tool to understand when and how they make mistakes.
This can help them improve all aspects of their play style~\cite{chess-com-chess-engines}.

Additionally, engines make it much easier to implement learning modules
and mini-games into a web app.
More advanced engines can also change skill level on demand.
This can be used for training the user as they can choose the difficulty of
their simulated opponent~\cite{chess-com-chess-engines}.
To summarize, engines are critical for modern chess web apps as they
expand how the web app can interact with its users.
This interaction will both increase usability, but also make guidance accessible
24/7 for users.

Engines make use of methods described in the following section to evaluate a chess position.
