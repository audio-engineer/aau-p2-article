\section{Multiplayer}\label{sec:multiplayer-analysis}

Chess is famously played by two players, where each player takes turns to move their pieces.
Because of this, the game is inherently a multiplayer game.
There are however different types of multiplayer games, and while chess is originally a tabletop game, adapting it to a
digital platform has an effect on the multiplayer aspect of the game.
This section will go over the different types of multiplayer games, and how the transition from local to online
multiplayer can have an effect on learning chess.

\subsection{Types of multiplayer games}\label{subsec:types-of-multiplayer-games}

Multiplayer is a term used to describe games that can be played by more than one player.
It is mostly associated with video games, but most tabletop games, such as chess, are also multiplayer games.
Throughout the years, there have been three dominating types of multiplayer implementations~\cite{multiplayer-types},
and chess can fall into two of these types.

\begin{itemize}

    \item \textbf{Local multiplayer} has the players take turns to play the game.
    It requires the players to be in the same room and to wait for their turn to play.
    Traditional tabletop games, such as chess, fall into this category.

    \item \textbf{Co-op multiplayer} has two players playing on a split-screen.
    It also requires the players to be in the same room, but unlike the first type, they can play at the same time.
    This type is mostly associated with video games, and it is not relevant for chess.

    \item \textbf{Online multiplayer} allows for players to play together over the internet, negating the requirement to
    be in the same room as with the previous two types.
    This type of multiplayer can be seen in online chess games.

\end{itemize}

It can be concluded that by adapting chess to a digital platform, its multiplayer changes from local to online
multiplayer.
While the game itself remains the same, the way players interact with each other changes drastically.
This can have an impact on the experience players have when learning and playing chess.

\subsection{Effects multiplayer has on learning}\label{subsec:effects-multiplayer-has-on-learning}

Online multiplayer has a lot of benefits over local multiplayer.
It makes it possible to connect with people from all over the world, as they don't have to be in the same room.
It also allows for people to play against strangers, usually in a competitive environment, fitting for chess.
However, it takes away from the social aspect of the game, as players rarely communicate with each other.

There are different arguments on the matter, one source praises how online multiplayer allows for people to connect and
socialize with people they never could before~\cite{multiplayer-online}, while another source complains about how
transitioning from local to online multiplayer has significantly impacted the social interaction with their
friend~\cite{multiplayer-local}.
Online multiplayer games often attempt to mitigate the lack of social interaction by implementing chat, voice or video
systems, but it is rare for players, especially strangers, to use these features.

This lack of communication can have a negative effect on learning chess, as it discourages players from reaching out
for help or advice.
That typically results in a poor experience for novices.
A lot of people would rather learn chess from friends, family or a coach, resulting in a local multiplayer environment.
Therefore, we can conclude that the transition from local to online multiplayer has a negative effect on the learning
experience of chess.
