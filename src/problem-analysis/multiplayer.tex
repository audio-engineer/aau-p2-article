% textidote: ignore begin
\section{Multiplayer analysis}\label{sec:multiplayer-analysis}
% textidote: ignore end

Chess is famously played by two players, where each player takes turns to move their pieces.
Because of this, the game is inherently a multiplayer game.
There are however different types of multiplayer games, and while chess is originally a tabletop game, adapting it to a 
digital platform has an effect on the multiplayer aspect of the game.

\subsection{Types of multiplayer games}\label{subsec:types-of-multiplayer-games}

Multiplayer is a term that has been made popular by the video game industry.
It has its roots in the arcade games and the early home consoles.
The term is used to describe games that are played by more than one player.
But the type of multiplayer games from back then is different from the multiplayer games of today.
Throughout the years, there have been three dominating types of multiplayer games~\cite{multiplayer-types}.

The first type has the players take turns to play the game.
It requires the players to be in the same room and to wait for their turn to play.
This type of game was mostly prevalent in the early days, where people would compete for a higher score.
It can be seen in not only video games, but also in tabletop games like chess.

The second type has two players playing on a split-screen.
It also requires the players to be in the same room, but unlike the first type, they can play at the same time.
Some variations of this multiplayer type allows the players to play on different screens.
This type of game is usually seen in co-op and competitive games, and it can still be seen today, but it is not as 
prevalent as it was in the past.

The third type allows for players to play together over the internet.
This type of multiplayer game is the most prevalent today, and it is the type that is most commonly associated with
multiplayer games.
As the name suggest, it allows for the players to play together over the internet, negating the requirement to be in the
same room as with the previous two types.
This type of multiplayer can often be seen in online chess games.

Therefore, it can be concluded that by adapting chess to a digital platform, the type of multiplayer game changes from 
the first type to the third type.
This can have a drastic effect of the experience players have when learning and playing chess.

\subsection{Effects multiplayer has on learning}\label{subsec:effects-multiplayer-has-on-learning}

From here on, this report will refer to the first type as local multiplayer and the third type as online 
multiplayer.
As previously discussed, chess as a tabletop is categorized as a local multiplayer game, while chess as a digital game 
is categorized as an online multiplayer game.

Online multiplayer has a lot of benefits over local multiplayer.
It makes it possible to connect with people from all over the world, as they don't have to be in the same room.
It also allows for people to play against strangers, usually in a competitive environment, fitting for chess.
However, it takes away from the social aspect of the game, as players rarely communicate with each other.

There are different arguments on the matter, one source praises how online multiplayer allows for people to connect and 
socialize with people they never could before~\cite{multiplayer-online}, while another source complains about how 
transitioning from local to online multiplayer has significantly impacted the social interaction with their 
friend~\cite{multiplayer-local}. 
Online multiplayer games often attempt to mitigate the lack of social interaction by implementing chat, voice or video 
systems, but it is rare for players, especially strangers, to use these features.

This lack of communication can have a negative effect on learning chess, as it discourages players from reaching out 
for help or advice.
That typically results in a poor experience for novices.
A lot of people would rather learn chess from friends, family or a coach, resulting in a local multiplayer environment.
Therefore, we can conclude that the transition from local to online multiplayer has a negative effect on the learning 
experience of chess.
