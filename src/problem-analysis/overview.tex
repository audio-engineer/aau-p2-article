% textidote: ignore begin
\section{Overview of the problem analysis}\label{sec:overview}

\textbf{Why Play games?}

\begin{itemize}
    \item Social activity.
    \item Fun.
    \begin{itemize}
        \item Although extremely broad every reason for playing a game ultimately leads to enjoyment or fun.
        Many different people enjoy many different games, some physical, some mental,
        but ultimately you play a game to have fun.
        This can be had either through the intricacies of the game itself,
        or because of the social interactions you get while playing games.
    \end{itemize}
\end{itemize}

\textbf{Chess and its genre}

\begin{itemize}
    \item Board game.
    \item Abstract strategy game.
    \item Mind sport.
\end{itemize}

\textbf{Why play chess?}

\begin{itemize}
    \item Playing to win
    \begin{itemize}
        \item When you win a game of chess you have outplayed
        your opponent which creates enjoyment.
        \item Competitive game.
    \end{itemize}
\end{itemize}

\textbf{Who plays chess?}

\begin{itemize}
    \item Non-competitive
    \begin{itemize}
        \item This Group of players enjoy chess as a game casually,
        they might not know all the intricacies of chess.
        \item The experience of this subsection of players range from: Beginner - competent
    \end{itemize}
    \item Competitive
    \begin{itemize}
        \item To play chess competitively you need to have a good understanding
        of the game and the underlying principles of how you win.
        \item The experience of this subsection ranges from competent - master level
    \end{itemize}
\end{itemize}

\textbf{State of the art}

\begin{itemize}
    \item Chess.com
    \begin{itemize}
        \item Play chess
        \begin{itemize}
            \item Different time modes
            \item different variations like, 4 player chess
            \item Tournaments
        \end{itemize}
        \item Learn chess
        \begin{itemize}
            \item Puzzles
            \item Playstyles
            \item Theory, openings and endgames.
            \item Analyze a game of chess using engines
        \end{itemize}
        \item Coaching
    \end{itemize}
    \item Lichess.org
    \begin{itemize}
        \item Although both Lichess.org and Chess.com are free in the beginning,
        chess.com limits functionality to sell a premium service,
        this is opposed to Lichess.org which is complete free,
        both in terms of cost, but the website is also open-source.
        \begin{itemize}
            \item You can also play chess and different variations on Lichess
            \item Lichess has a more comprehensive learning section.
            \item Coaching
        \end{itemize}
    \end{itemize}
\end{itemize}

\textbf{Weaknesses in the state of the art}

\begin{itemize}
    \item As mentioned above, when learning chess you have ample opportunity for learning through puzzles analysis,
    but what all these things have in common, or lack is “human contact”.
    When learning chess the only viable options are studying or learning with a computer.
    This leaves an opening in finding some way to incorporate learning chess specifically while playing with someone
    else and having the social aspect that is more characteristic of non-competitive chess.
\end{itemize}

\textbf{Target audience}

\begin{itemize}
    \item The product mentioned above would be some way to learn chess
    while playing with someone else that has a learning aspect incorporated into the chess game beyond simply playing.
    \begin{itemize}
        \item As you will receive help this is not a website build for competitive use.
        \item The implementation could be allowing some level of engine help while playing, and because chess
        engines are at a level far beyond human capabilities, the ceiling for learning through this method is unclear.
    \end{itemize}
\end{itemize}
% textidote: ignore end
