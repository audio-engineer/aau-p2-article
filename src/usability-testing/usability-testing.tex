\chapter{Usability Testing}\label{ch:user-test}

After the development of the website and the game,
the team conducted usability testing to gather feedback on the project.
This feedback will be used to improve the website and game before the final release.
It can also provide the team with insights into the user experience and the game's playability.

\section{Methodology}\label{sec:methodology}

The tests are conducted by multiple members of the team.
To keep the results consistent, a methodology was established.
It consists of a script that the team members follow, ensuring that the same questions were asked to each participant.
The script is divided into four parts: introduction, user interface and experience, gameplay, and feedback.
All are designed like an interview, where the team members will first show them the aspects of the website,
then let them play the game for themselves, and finally ask a series of questions to gather their feedback.
The results in Section~\ref{sec:results} are based on their feedback.

\section{Participants}\label{sec:participants}

A total of three participants were recruited for the usability testing.
They varied between age and skill level, with at least two of them fitting the target users of the game.
This section will introduce the participants and their background.

\begin{itemize}

    \item \textbf{Ronja} is 22 years old, and she ranks at 800 ELO, which fits in the target users of beginners.
    She was taught to play in her kindergarten, and now she plays chess occasionally for fun.

    \item \textbf{Clara} is 19 years old, and she ranks at 1000 ELO, which also fits in the target users.
    She studies machine learning, so she could provide valuable feedback on the Stockfish implementation.
    She was taught to play chess in an academic setting.

    \item \textbf{Peter} is 31 years old, and his rank is 1700 ELO, which is above the target users.
    He plays chess semi-professionally, first learning to play from his father and then in school.
    % textidote: ignore begin % due to chess.com.
    He got more interested in chess during the pandemic due to Chess.com's popularity and chess' presence on Twitch.
    % textidote: ignore end
    While he may not be the target users range, his experience could provide the team with valuable feedback.

\end{itemize}

% textidote: ignore begin % due to section being too short
\section{Results}\label{sec:results}
% textidote: ignore end

The results of the test are based on the feedback from the participants.
It is compacted into two main focus points which are the learning aspect of the project and a potential solution to
their feedback.
The last subsection will cover any other feedback that the participants provided.

\subsection{Learning experience}\label{subsec:learning-experience}

The main focus point with the usability testing was the learning aspect of the project, because that's the goal of the
project.
The feedback on the learning aspect was not very positive.
Ronja didn't immediately understand what the Stockfish feedback was, but after using Stockfish a little, she lost
interest in it.
Clara thought that the Stockfish feedback was too direct, and Peter mentioned that the feedback should provide future
moves along with the current move.
None of them thought that the feedback was helpful.
This indicates that the feedback from Stockfish has a lot of room for improvement, as only providing the best move is
not enough for learning, because players wouldn't understand why it's the best move.
The team was already aware of this issue and had discussed a potential solution, which was conveyed to the users in the
next focus point.

\subsection{Potential solution}\label{subsec:potential-solution}

As a potential solution, users were asked about the idea of using a large language model along with Stockfish to provide
feedback.
The rationale behind this question was due to a prior discussion within the team, which will be later discussed in
Section~\ref{sec:large-language-model-implementation}.
Clara and Ronja both liked the idea of using a large language model, while Peter was more skeptical.
He did not particularly like the idea due to the lack of control the developers would have over the large language
model's answers.
This will make it such that the team cannot be guaranteed that it will provide the correct information.
Clara, however, thinks that in combination with Stockfish it could provide accurate information whilst also giving more
natural responses and explanations.
She and Ronja think that a natural response would help a lot with understanding the feedback.
An example given was that the player could ask follow-up questions, which the large language model could answer,
providing a more interactive learning experience.

\subsection{Other feedback}\label{subsec:other-feedback}

The team got a lot of feedback unrelated to Stockfish that can also hinder the learning experience.
Peter pointed out that the front page was not intuitive, and Clara suggested that the buttons should be larger, labeled,
and link together.
Peter also mentioned that it was challenging to figure out how to play, saying that the About page should contain
more technical information.
Clara noted that the Stockfish feedback lacked visual cues, which is something that the team had also previously
discussed.
Peter mentioned that the notations provided by Stockfish were overcomplicated, complained about the lack of sound
effects and no clear indication for when a game is over.
Clara mentioned that the game should have more teaching tools, such as puzzles and guides to how to play chess.
The team considered these features given the learning nature of the game, but they focused on the coaching aspect of the
game instead.

\section{Testing Conclusion}\label{sec:tests-conclusion}

While there are a lot of issues that need to be addressed, the overall opinion of the project was positive.
At the end of the usability testing, the participants were asked for their final thoughts on the project.
Clara and Ronja had a positive experience, and they would use the website again because they like the idea of using an
engine like Stockfish to improve their chess skills.
Peter, on the other hand, would rather use alternatives like~\url{Chess.com} and Lichess, because he believes that they
can do everything our website can do and more.

The main point of concern was the Stockfish feedback, because that's the main feature of the project.
The team has already discussed a potential solution to this issue, which was well received by the participants.
There was also a lot of great feedback that the team was not aware of, which will be taken into consideration when
improving the website and game.
However, since the focus of the project is on the learning aspect, the main priority will be to improve the Stockfish
feedback.
Two of the three participants like the idea of using a large language model to provide feedback, so the team is more
confident in this solution.
Overall, the usability testing was a success, and the feedback provided valuable insights for the team on how to improve
the website and game before the final release.
