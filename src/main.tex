% Preamble
\documentclass[11pt]{report}

% Packages
\usepackage[a4paper]{geometry}
\usepackage[english]{babel}
\usepackage[backend=biber,style=ieee]{biblatex}
\usepackage{hyphenat}
\usepackage{csquotes}
\usepackage[indent=20pt]{parskip}
\usepackage{hyperref}
\usepackage{graphicx}
\usepackage{listings}
\usepackage{subfiles}
\usepackage[table,xcdraw]{xcolor}
\usepackage[normalem]{ulem}
\usepackage{colortbl}
\usepackage{float}
\usepackage{booktabs}
\usepackage{tcolorbox}
\usepackage{aaufrontmatter}

% textidote: ignore begin
\useunder{\uline}{\ul}{}
% textidote: ignore end

% textidote: ignore begin

% Bibliography
\addbibresource{problem-analysis/chess.bib}
\addbibresource{problem-analysis/state-of-the-art/SOTA.bib}
\addbibresource{problem-analysis/target-group-analysis.bib}
\addbibresource{problem-analysis/learning.bib}
\addbibresource{problem-analysis/relevance.bib}
\addbibresource{method/technology.bib}
\addbibresource{problem-solution/software-req-spec.bib}
\addbibresource{problem-solution/backend.bib}
\addbibresource{process-analysis/process-analysis.bib}
\addbibresource{discussion/llm-implementation.bib}

% Configuration
\graphicspath{ {./images/} }

\hypersetup{pdfborder=0 0 0}

% textidote: ignore end

% Document
\title{Conquer the Chessboard}
\subtitle{A Beginner's Journey with AI Coaching}
\theme{Software Engineering}
\semester{Spring}
\groupnumber{8}
\groupmembers{
    Kristiyan Mariyan Georgiev,
    Mads Heilmann,
    Martin Kedmenec,
    Sebastian Hemmer Bech Mygind,
    Simon Woidemann
}
\supervisors{Andres Masegosa}
\department{Software}
\abstract{This project aims to facilitate learning of the game chess by replacing the traditional coach with a
chess engine.
The key focus points of the report are the learning aspect of chess, primarily focusing on why assisted
learning, such as coaching, is more effective than individual learning.
The outcome is an online multiplayer game called ChessTeacher, where two people can play against each other, while
receiving coaching from a chess engine in the form of feedback on the players' moves, and suggest improvements.
The results the authors found during this project indicate that using chess engines for coaching is viable, but it is
lacking in terms of communicating with the players.
A promising addition to chess engine coaching is LLMs, as they could make up for chess engines'
lack of communication skills.
This would help mitigate the biggest issue the test users found while testing the app.}

\begin{document}
    % Title
    \maketitle

    % Table of Contents
    % textidote: ignore begin
    \tableofcontents
    % textidote: ignore end

    % Commenting out until it's ready
    % \makeaaupreface{This is the preface.}

    % Main Content
    \input{codestyling/codestyling}
    % textidote: ignore begin
\chapter{Introduction}\label{ch:introduction}
% textidote: ignore end

Learning is something we do every day.
It is something most people enjoy and strive to experience.
However, learning can also be difficult.
In this report we will explore the concept of collaborated assisted learning and how software can assist in this method.
We will be focusing our analysis around the age-old game of chess.

Chess has seen a rise in player count in recent years, and with the growing player base the current chess websites have
perfected their craft, this has been done through intuitive design and the time to implement extra features in the apps.
One reason that chess has become more popular online, is that using the popular web apps like \url{chess.com} or
\url{lichess.org} allows the player to use a powerful tool to aid in their learning journey, that is chess engines.

Chess engine popularity has risen greatly in the past years and have been developed to a point that they
outperform humans.
These tools are essential to understanding the complexity of the end- and mid-game strategies of chess.
Chess encapsulates the spirit of a game that is easy to learn, but hard to master.
We will be focusing on the learning aspect of chess by making a chess web app, that allows two players to play against
each other with a coach.
This coach will be a chess engine, which will allow both players to learn while playing in a multiplayer environment.

Through this approach we touch a niche that lies between the social aspect of a player vs player chess game and the
great learning tools that are made to study chess and its complexities individually.
The product we are suggesting would then take the best of both worlds, by targeting the social aspects of the game while
still providing useful information to the player on how to improve their chess capabilities.
We go into more detail in Section~\ref{sec:state-of-the-art}.

    % textidote: ignore begin
\chapter{Problem analysis}\label{ch:problem-analysis}
% textidote: ignore end

% textidote: ignore begin
\section{Overview of the problem analysis}\label{sec:overview}

\textbf{Why Play games?}

\begin{itemize}
    \item Social activity.
    \item Fun.
    \begin{itemize}
        \item Although extremely broad every reason for playing a game ultimately leads to enjoyment or fun.
        Many different people enjoy many different games, some physical, some mental,
        but ultimately you play a game to have fun.
        This can be had either through the intricacies of the game itself,
        or because of the social interactions you get while playing games.
    \end{itemize}
\end{itemize}

\textbf{Chess and its genre}

\begin{itemize}
    \item Board game.
    \item Abstract strategy game.
    \item Mind sport.
\end{itemize}

\textbf{Why play chess?}

\begin{itemize}
    \item Playing to win
    \begin{itemize}
        \item When you win a game of chess you have outplayed
        your opponent which creates enjoyment.
        \item Competitive game.
    \end{itemize}
\end{itemize}

\textbf{Who plays chess?}

\begin{itemize}
    \item Non-competitive
    \begin{itemize}
        \item This Group of players enjoy chess as a game casually,
        they might not know all the intricacies of chess.
        \item The experience of this subsection of players range from: Beginner - competent
    \end{itemize}
    \item Competitive
    \begin{itemize}
        \item To play chess competitively you need to have a good understanding
        of the game and the underlying principles of how you win.
        \item The experience of this subsection ranges from competent - master level
    \end{itemize}
\end{itemize}

\textbf{State of the art}

\begin{itemize}
    \item Chess.com
    \begin{itemize}
        \item Play chess
        \begin{itemize}
            \item Different time modes
            \item different variations like, 4 player chess
            \item Tournaments
        \end{itemize}
        \item Learn chess
        \begin{itemize}
            \item Puzzles
            \item Playstyles
            \item Theory, openings and endgames.
            \item Analyze a game of chess using engines
        \end{itemize}
        \item Coaching
    \end{itemize}
    \item Lichess.org
    \begin{itemize}
        \item Although both Lichess.org and Chess.com are free in the beginning,
        chess.com limits functionality to sell a premium service,
        this is opposed to Lichess.org which is complete free,
        both in terms of cost, but the website is also open-source.
        \begin{itemize}
            \item You can also play chess and different variations on Lichess
            \item Lichess has a more comprehensive learning section.
            \item Coaching
        \end{itemize}
    \end{itemize}
\end{itemize}

\textbf{Weaknesses in the state of the art}

\begin{itemize}
    \item As mentioned above, when learning chess you have ample opportunity for learning through puzzles analysis,
    but what all these things have in common, or lack is “human contact”.
    When learning chess the only viable options are studying or learning with a computer.
    This leaves an opening in finding some way to incorporate learning chess specifically while playing with someone
    else and having the social aspect that is more characteristic of non-competitive chess.
\end{itemize}

\textbf{Target audience}

\begin{itemize}
    \item The product mentioned above would be some way to learn chess
    while playing with someone else that has a learning aspect incorporated into the chess game beyond simply playing.
    \begin{itemize}
        \item As you will receive help this is not a website build for competitive use.
        \item The implementation could be allowing some level of engine help while playing, and because chess
        engines are at a level far beyond human capabilities, the ceiling for learning through this method is unclear.
    \end{itemize}
\end{itemize}
% textidote: ignore end

% textidote: ignore begin
\section{Chess}\label{sec:chess}
% textidote: ignore end

Chess is a game that has existed for millennia.
It has roots in India as well as China where the first iterations of chess we know today were
invented~\cite{murray1913}.

Nowadays, the chess standard universally known as chess is the iteration of the game called international chess or
western chess, which is also the standard that FIDE, the international chess governing body, uses~\cite{fide2024}.

Most people are familiar with chess, hereby the rules of the game and the look of the board.
Specific notations and additional rules that are not common knowledge will be explained later on
when they become relevant to better understand this report.

% textidote: ignore begin
\section{Multiplayer analysis}\label{sec:multiplayer-analysis}
% textidote: ignore end

Chess is famously played by two players, where each player takes turns to move their pieces.
Because of this, the game is inherently a multiplayer game.
There are however different types of multiplayer games, and while chess is originally a tabletop game, adapting it to a 
digital platform has an effect on the multiplayer aspect of the game.

\subsection{Types of multiplayer games}\label{subsec:types-of-multiplayer-games}

Multiplayer is a term that has been made popular by the video game industry.
It has its roots in the arcade games and the early home consoles.
The term is used to describe games that are played by more than one player.
But the type of multiplayer games from back then is different from the multiplayer games of today.
Throughout the years, there have been three dominating types of multiplayer games~\cite{multiplayer-types}.

The first type has the players take turns to play the game.
It requires the players to be in the same room and to wait for their turn to play.
This type of game was mostly prevalent in the early days, where people would compete for a higher score.
It can be seen in not only video games, but also in tabletop games like chess.

The second type has two players playing on a split-screen.
It also requires the players to be in the same room, but unlike the first type, they can play at the same time.
Some variations of this multiplayer type allows the players to play on different screens.
This type of game is usually seen in co-op and competitive games, and it can still be seen today, but it is not as 
prevalent as it was in the past.

The third type allows for players to play together over the internet.
This type of multiplayer game is the most prevalent today, and it is the type that is most commonly associated with
multiplayer games.
As the name suggest, it allows for the players to play together over the internet, negating the requirement to be in the
same room as with the previous two types.
This type of multiplayer can often be seen in online chess games.

Therefore, it can be concluded that by adapting chess to a digital platform, the type of multiplayer game changes from 
the first type to the third type.
This can have a drastic effect of the experience players have when learning and playing chess.

\subsection{Effects multiplayer has on learning}\label{subsec:effects-multiplayer-has-on-learning}

From here on, this report will refer to the first type as local multiplayer and the third type as online 
multiplayer.
As previously discussed, chess as a tabletop is categorized as a local multiplayer game, while chess as a digital game 
is categorized as an online multiplayer game.

Online multiplayer has a lot of benefits over local multiplayer.
It makes it possible to connect with people from all over the world, as they don't have to be in the same room.
It also allows for people to play against strangers, usually in a competitive environment, fitting for chess.
However, it takes away from the social aspect of the game, as players rarely communicate with each other.

There are different arguments on the matter, one source praises how online multiplayer allows for people to connect and 
socialize with people they never could before~\cite{multiplayer-online}, while another source complains about how 
transitioning from local to online multiplayer has significantly impacted the social interaction with their 
friend~\cite{multiplayer-local}. 
Online multiplayer games often attempt to mitigate the lack of social interaction by implementing chat, voice or video 
systems, but it is rare for players, especially strangers, to use these features.

This lack of communication can have a negative effect on learning chess, as it discourages players from reaching out 
for help or advice.
That typically results in a poor experience for novices.
A lot of people would rather learn chess from friends, family or a coach, resulting in a local multiplayer environment.
Therefore, we can conclude that the transition from local to online multiplayer has a negative effect on the learning 
experience of chess.

% textidote: ignore begin
\section{Learning methods for chess}\label{sec:learning-methods-for-chess}
% textidote: ignore end

When learning, a lot of different methodologies and techniques can be applied.
Mentioned is the core elements of our learning scheme to be imposed on the subject.

% textidote: ignore begin
\subsection{Trial \& error}\label{subsec:trial-and-error}
% textidote: ignore end

One learning method is the trial and error methodology.
For this way of learning to be effective the subject, i.e.\ the student, needs to be able to make mistakes and receive
instant feedback about either the consequences or the benefits of the action~\cite{li2023}.
In the case of chess, the student must be able to make mistakes.
These mistakes can range from short term mistakes, blunders, to long term mistakes such as positional mistakes.
In addition, the rewards for performing good moves needs to encapsulate the reason for it being a good move, such that
the student understand the benefits and can compare their thought process to that of the feedback.

% textidote: ignore begin
\subsection{Assisted learning}\label{subsec:assisted-learning}
% textidote: ignore end

Studies suggest that peer-assisted learning is more effective than individual learning at higher task complexity and
difficulty~\cite{carson2023}.
Chess is arguably a complex game.
One argument for its complexity is that chess remains an `unsolved' game for the most part.
For a game to be solved all known best moves for all possible positions of the game needs to be identified or be
identifiable relatively easily through algorithms or the like~\cite{herik2002}.
An example of a solved game is checkers~\cite{schaeffer2007}.
Chess with six or fewer pieces left on the board including the kings is solved and encapsulates roughly only
\( 4 \times 10^{12} \) positions out of all possible positions~\cite{syzygy2024}.
However, chess has an upper bound of \( 2 \times 10^{40} \) different legal positions, not counting positions where one
or more pawns has promoted~\cite{steinerberger2014}.
This extraordinary large amount of possible positions makes for a greatly challenging game to learn.
It can hereby be observed that peer-assisted learning will likely have a greater impact on individuals learning chess
than individuals learning chess by themselves.

Another factor that improves the effectiveness of learning is the ability to interact with educators and to ask
follow-up questions about a desired topic.
Hereby it is also important to have a good relationship with one's educators, where both parties feel comfortable with
each other and where a comfortable learning environment is constructed~\cite{saha2009}.
In the context of an online non-person educator, the program should be able to adapt and relate to the student to
produce a more effective learning environment for the student.

To be able to efficiently and effectively learn, one should not only remember but also understand what they are taught.
Explanations are vital for this understanding and is necessary for students to learn effectively~\cite{williams2010}.
In the context of the ChessTeacher project, the program must be able to provide detailed case specific explanations to
the user.

\section{Target group analysis}\label{sec:target-group-analysis}

Chess as a game has a broad appeal, as it is played by many people, young and old.
As previously stated, chess has risen in popularity recently, so it is sufficed to say that the user group is a lot
larger than it was a few years ago.
Therefore, it is necessary to analyze the target group to narrow down the scope of the problem.

A good way to do just that is by splitting the target group into different categories.
It is possible to split the people that play chess by age, skill level, and how frequently they play.
As the project is about the learning process of chess, the most relevant choice is to divide the target group based on
skill level.
To measure their skill level, the team uses the ELO rating system, which is a method for calculating the relative skill
levels of players in two-player games such as chess, using data from~\url{Chess.com}.

% textidote: ignore begin
\begin{figure}[H]
    \centering
    \includegraphics[width=1\textwidth]{chess-graph}
    \caption{Rating distribution graph based off of Chess.com's rating system~\cite{chess-ratings}.}\label{fig:graph}
\end{figure}
% textidote: ignore end

Figure~\ref{fig:graph} shows a graph made to illustrate different skill levels amongst the user base
of~\url{Chess.com}.
The blue wave represents the number of users with a certain rating.
The team doesn't have direct numbers as the graph is very limited in terms of data.
The color bars on the bottom represent different skill groups~\cite{chess-ratings}.

The first group, which is marked in green and is the biggest one, is novices.
They are not expected to know much about chess or its rules.
This is why they are not fit to be the target group, as the goal is not to teach the rules of chess, but rather to help
people improve their skills.
The second group, which is marked in yellow, is beginners.
This is the main target group, as they are the ones that are expected to benefit the most from the project.
That's because the graph peaks at the end of the novices group, which means that transitioning from novice to beginner
seems to be the hardest challenge for new players.
The team has therefore decided to focus on this group; to help them overcome this obstacle.
By focusing on this group, the team hopes to flatten the curve and make the transition from beginner to intermediate
easier.
The third and fourth groups, which are marked in light and dark orange, are intermediate one and two players.
While they are not the main target group, they may still have a subset of the problems that the beginners have.
Therefore, they can still benefit from the project.
The last group, which is marked in red, is expert players.
Unlike intermediate players, they are less likely to have the issues that the beginners have.

The yellow line shows the cumulative percentage of the data as the graph approaches the right.
It signals that the target group encapsulates all users between the 45th and the 65th percentile, which corresponds to
at least 20\% of blitz users on~\url{Chess.com}~\cite{chess-ratings}.

% textidote: ignore begin
\section{State of the art}\label{sec:state-of-the-art}
% textidote: ignore end

The online chess web application landscape is a vibrant space with several popular
solutions.
Two solutions, \url{chess.com} and \url{lichess.org} stand out as leaders in terms
of both user base and feature completeness for learning and playing chess.
This section will delve into these two prominent platforms, while providing a
comparative analysis on the features the authors find to be what makes a
chess application a success.

% textidote: ignore begin
\subsection{Chess.com}\label{subsec:chess-com}
% textidote: ignore end
Hello chess.

% textidote: ignore begin
\subsection{Lichess.org}\label{subsec:lichess-org}
% textidote: ignore end

\url{Lichess.org} is a free, open-source online chess platform that offers a variety of services to its users.
Lichess was founded in 2010 by Thibault Duplessis, and has since then grown to be one of the most popular chess websites
in the world.

Unlike~\url{Chess.com},~\url{Lichess.org} is completely free and open-source meaning that the website is free to use,
and the source
code is available for anyone to use and modify.
This makes~\url{Lichess.org} a unique platform in the chess community.
Unlike a lot of other chess websites, they do not charge users for their services, while also being an open-source
platform where the community can contribute to the development of the website.

\subsubsection{What does Lichess offer?}\label{subsubsec:what-does-lichess-offer}

As~\url{chess.com},~\url{Lichess.org} offers a wide range of services to its users, such as playing chess,
learning chess, and coaching.
On~\url{Lichess.org} you can train your skills by playing against a computer or by simply solving puzzles.
\url{Lichess.org} is also great for learning chess, as it offers almost the same features as \url{chess.com}
such as lessons, puzzles and articles and more.

\url{Lichess.org}, as~\url{Chess.com}, has a way of analyzing games.
Both~\url{Chess.com} and~\url{Lichess.org} use Stockfish as their analysis engine, which makes players able to
refine their skills.
Both Lichess and~\url{Chess.com} also offer a mobile application for their services, which is a great feature for
players who are on the go.

It also has a section for studying new tactics and openings called “Study” where you can create your own studies for
a new opening or strategy.
Lichess focuses more on the community aspect of chess.
As said before, it is a platform for chess players of all skills to share information about chess strategies,
tactics and more.
%textidote: ignore begin
The Lichess community is very active and has a large number of players who are active in forums and chat rooms.
%textidote: ignore end
They host regular tournaments for players to compete in, and like~\url{chess.com} the skill level can vary
from casual to championship level tournaments.

% textidote: ignore begin
\subsection{Comparative analysis}\label{subsec:comparative-analysis}
% textidote: ignore end

Reference is made to Table~\ref{tab:comparative-analysis} for the differences and similarities in the state-of-the-art
solutions.
Explanations for the services and features can be found under the table.

In the table, it can be observed that OTB chess entails the possibility of the players learning from each other and
players playing multiplayer chess when offline.
This potentially makes it one of the best ways to learn chess offline, whilst being able to play multiplayer and having
interactive learning simultaneously.
It can also be observed that the market gap will not be able to encompass offline users, as it will be an online web
application.
However, in the online section, it is observed that only chess educators and the market gap provide the multiplayer
learning experience to the user.
The downside of knowledgeable educators is that they charge money for their teaching service, why the market gap will be
the only solution that can potentially satisfy all the features in the table.
In the table is not included all the other services that~\url{Lichess.org} and~\url{Chess.com} provide, as that is not
the focus of the report.

% textidote: ignore begin
\begin{table}[H]
    \centering
    \resizebox{\columnwidth}{!}{%
        \begin{tabular}{clllll}
            \toprule
            \rowcolor[HTML]{9B9B9B}
            \textbf{Features} &
            \multicolumn{5}{c}{\cellcolor[HTML]{9B9B9B}\textbf{Services}}
            \\ \midrule
            \rowcolor[HTML]{EFEFEF}
            \cellcolor[HTML]{C0C0C0}{\color[HTML]{000000} \textit{\textbf{Online}}} &
            \multicolumn{1}{c}{\cellcolor[HTML]{EFEFEF}\textbf{Chess.com}} &
            \multicolumn{1}{c}{\cellcolor[HTML]{EFEFEF}\textbf{Lichess.com}} &
            \multicolumn{1}{c}{\cellcolor[HTML]{EFEFEF}\textbf{OTB chess}} &
            \multicolumn{1}{c}{\cellcolor[HTML]{EFEFEF}\textbf{Chess Educator}} &
            \multicolumn{1}{c}{\cellcolor[HTML]{EFEFEF}\textbf{Market gap}}
            \\ \midrule
            \rowcolor[HTML]{67FD9A}
            \cellcolor[HTML]{EFEFEF}\textbf{Multiplayer} &
            \multicolumn{1}{l}{\cellcolor[HTML]{67FD9A}} &
            \multicolumn{1}{l}{\cellcolor[HTML]{67FD9A}} &
            \multicolumn{1}{l}{\cellcolor[HTML]{FD6864}} &
            \multicolumn{1}{l}{\cellcolor[HTML]{67FD9A}} &
            \\ \midrule
            \rowcolor[HTML]{67FD9A}
            \cellcolor[HTML]{EFEFEF}\textbf{Learning} &
            \multicolumn{1}{l}{\cellcolor[HTML]{67FD9A}} &
            \multicolumn{1}{l}{\cellcolor[HTML]{67FD9A}} &
            \multicolumn{1}{l}{\cellcolor[HTML]{FD6864}} &
            \multicolumn{1}{l}{\cellcolor[HTML]{67FD9A}} &
            \\ \midrule
            \rowcolor[HTML]{FD6864}
            \cellcolor[HTML]{EFEFEF}\textbf{Multiplayer learning} &
            \multicolumn{1}{l}{\cellcolor[HTML]{FD6864}} &
            \multicolumn{1}{l}{\cellcolor[HTML]{FD6864}} &
            \multicolumn{1}{l}{\cellcolor[HTML]{FD6864}} &
            \multicolumn{1}{l}{\cellcolor[HTML]{67FD9A}} &
            \cellcolor[HTML]{67FD9A}
            \\ \midrule
            \rowcolor[HTML]{67FD9A}
            \cellcolor[HTML]{EFEFEF}\textbf{Chess engine} &
            \multicolumn{1}{l}{\cellcolor[HTML]{67FD9A}} &
            \multicolumn{1}{l}{\cellcolor[HTML]{67FD9A}} &
            \multicolumn{1}{l}{\cellcolor[HTML]{FD6864}} &
            \multicolumn{1}{l}{\cellcolor[HTML]{FD6864}} &
            \\ \midrule
            \rowcolor[HTML]{67FD9A}
            \cellcolor[HTML]{EFEFEF}\textbf{Free to use} &
            \multicolumn{1}{l}{\cellcolor[HTML]{67FD9A}} &
            \multicolumn{1}{l}{\cellcolor[HTML]{67FD9A}} &
            \multicolumn{1}{l}{\cellcolor[HTML]{67FD9A}} &
            \multicolumn{1}{l}{\cellcolor[HTML]{FD6864}} &
            \\ \midrule
            \rowcolor[HTML]{EFEFEF}
            \cellcolor[HTML]{C0C0C0}{\color[HTML]{333333} \textit{\textbf{Local}}} &
            \multicolumn{1}{c}{\cellcolor[HTML]{EFEFEF}\textbf{Chess.com}} &
            \multicolumn{1}{c}{\cellcolor[HTML]{EFEFEF}\textbf{Lichess.com}} &
            \multicolumn{1}{c}{\cellcolor[HTML]{EFEFEF}\textbf{OTB chess}} &
            \multicolumn{1}{c}{\cellcolor[HTML]{EFEFEF}\textbf{Chess Educator}} &
            \multicolumn{1}{c}{\cellcolor[HTML]{EFEFEF}\textbf{Market gap}}
            \\ \midrule
            \rowcolor[HTML]{FD6864}
            \cellcolor[HTML]{EFEFEF}\textbf{Multiplayer} &
            \multicolumn{1}{l}{\cellcolor[HTML]{67FD9A}} &
            \multicolumn{1}{l}{\cellcolor[HTML]{67FD9A}} &
            \multicolumn{1}{l}{\cellcolor[HTML]{67FD9A}} &
            \multicolumn{1}{l}{\cellcolor[HTML]{67FD9A}} &
            \\ \midrule
            \rowcolor[HTML]{67FD9A}
            \cellcolor[HTML]{EFEFEF}\textbf{Learning} &
            \multicolumn{1}{l}{\cellcolor[HTML]{67FD9A}} &
            \multicolumn{1}{l}{\cellcolor[HTML]{67FD9A}} &
            \multicolumn{1}{l}{\cellcolor[HTML]{67FD9A}} &
            \multicolumn{1}{l}{\cellcolor[HTML]{67FD9A}} &
            \cellcolor[HTML]{FD6864}
            \\ \midrule
            \rowcolor[HTML]{FD6864}
            \cellcolor[HTML]{EFEFEF}\textbf{Multiplayer learning} &
            \multicolumn{1}{l}{\cellcolor[HTML]{FD6864}} &
            \multicolumn{1}{l}{\cellcolor[HTML]{FD6864}} &
            \multicolumn{1}{l}{\cellcolor[HTML]{67FD9A}} &
            \multicolumn{1}{l}{\cellcolor[HTML]{67FD9A}} &
            \\ \midrule
            \rowcolor[HTML]{FD6864}
            \cellcolor[HTML]{EFEFEF}\textbf{Chess engine} &
            \multicolumn{1}{l}{\cellcolor[HTML]{67FD9A}} &
            \multicolumn{1}{l}{\cellcolor[HTML]{67FD9A}} &
            \multicolumn{1}{l}{\cellcolor[HTML]{FD6864}} &
            \multicolumn{1}{l}{\cellcolor[HTML]{FD6864}} &
            \\ \midrule
            \rowcolor[HTML]{67FD9A}
            \cellcolor[HTML]{EFEFEF}\textbf{Free to use} &
            \multicolumn{1}{l}{\cellcolor[HTML]{67FD9A}} &
            \multicolumn{1}{l}{\cellcolor[HTML]{67FD9A}} &
            \multicolumn{1}{l}{\cellcolor[HTML]{67FD9A}} &
            \multicolumn{1}{l}{\cellcolor[HTML]{FD6864}} &
            \\ \bottomrule
        \end{tabular}%
    }
    \caption{Comparative analysis of the services and some features they have.
    A green color indicates that the service encapsulates the feature.
    A red color indicates that the service does not encapsulate the feature.}\label{tab:comparative-analysis}
\end{table}
% textidote: ignore end

\begin{itemize}
    % textidote: ignore begin % to keep chess.com and lichess.org consistent with the other items
    \item \textbf{{Chess.com}}: The chess website as explained in Section~\ref{subsec:chess-com}.
    \item \textbf{{Lichess.org}}: The chess website as explained in Section~\ref{subsec:lichess-org}.
    % textidote: ignore end
    \item \textbf{OTB chess}: Short for `over the board chess',
    which is the sport of physical chess, where one is accompanied by a psychical chess board.
    \item \textbf{Chess Educator}: The service where an actual physical person is accompanying the user, either online
    or offline, and giving the user feedback and helping vocally by communicating their thoughts about the users moves
    and the game to the user.
    \item \textbf{Market gap}: Encapsulates the solution we produce with the functionalities described in
    Section~\ref{sec:software-requirement-specification}.
    \item \textbf{Multiplayer}: The ability to play with another person either online or offline.
    \item \textbf{Learning}: The ability to achieve strategic knowledge about the game of chess through various learning
    methods.
    \item \textbf{Multiplayer learning}: A combination of the two beforehand mentioned features, with the modification
    that it can be achieved simultaneously.
    \item \textbf{Chess engine}: The existence of an integrated chess engine as they are described in
    Section~\ref{sec:chess-engines}.
    \item \textbf{Free to use}: The ability to use the service without it costing the user any money.
\end{itemize}



    % textidote: ignore begin
\chapter{Problem statement}\label{ch:problem-statement}
% textidote: ignore end

% textidote: ignore begin
\section{Problem delineation}\label{sec:problem-delineation}
% textidote: ignore end

% textidote: ignore begin
\section{Problem statement}\label{sec:problem-statement}
% textidote: ignore end

How can we accelerate learning chess for beginners using software?

    \input{method/method}
    % textidote: ignore begin
\chapter{Problem solution}\label{ch:problem-solution}
% textidote: ignore end

% textidote: ignore begin
\section{Software requirement specification}\label{sec:software-requirement-specification}
% textidote: ignore end

To answer to problem statement, an outline of the desired program features is made.
The method of choice for the requirement specifications is the MoSCoW model~\cite{hudaib2018}.
The overview of the features the application will and will not entail are provided in the following Subsections
~\ref{subsec:must-haves},~\ref{subsec:should-haves},~\ref{subsec:could-haves} and~\ref{subsec:wont-haves}.

% textidote: ignore begin
\subsection{Must-haves (Mo)}\label{subsec:must-haves}
% textidote: ignore end

\textbf{``Must-haves''} encapsulates all the features that must be
included in the final product.

\begin{itemize}
    \item Basic chess functionality
    \item Multiplayer functionality
    \item Applied learning principles
    \item Chess engine implementation
\end{itemize}

% textidote: ignore begin
\subsection{Should-haves (S)}\label{subsec:should-haves}
% textidote: ignore end

\textbf{``Should-haves''} encapsulates features that are desired and that we are technically able to implement if not
for the time constraints of the project.

\begin{itemize}
    \item User accounts
    \item User registration and authentication
    \item Engine feedback in English
    \item Chat functionality
    \item Server listing
\end{itemize}

% textidote: ignore begin
\subsection{Could-haves (Co)}\label{subsec:could-haves}
% textidote: ignore end

\textbf{``Could-haves''} encapsulates features that are possible
to implement but are not the focus of this report.

\begin{itemize}
    \item Analysis tool for post game analysis
    \item Chess puzzles
    \item Achievements
    \item Board customization
\end{itemize}

% textidote: ignore begin
\subsection{Won't-haves (W)}\label{subsec:wont-haves}
% textidote: ignore end

\textbf{``Won't-haves''} encapsulates out of scope features that are not going to be considered or implemented.

\begin{itemize}
    \item AI vs AI matches
    \item Chess variants other than standard chess
    \item AI training and machine learning
\end{itemize}


    \chapter{Usability Testing}\label{ch:user-test}

After the development of the website and the game,
the team conducted usability testing to gather feedback on the project.
This feedback will be used to improve the website and game before the final release.
It can also provide the team with insights into the user experience and the game's playability.

\section{Methodology}\label{sec:methodology}

The tests are conducted by multiple members of the team.
To keep the results consistent, a methodology was established.
It consists of a script that the team members follow, ensuring that the same questions were asked to each participant.
The script is divided into four parts: introduction, user interface and experience, gameplay, and feedback.
All are designed like an interview, where the team members will first show them the aspects of the website,
then let them play the game for themselves, and finally ask a series of questions to gather their feedback.
The results in Section~\ref{sec:results} are based on their feedback.

\section{Participants}\label{sec:participants}

A total of three participants were recruited for the usability testing.
They varied between age and skill level, with at least two of them fitting the target users of the game.
This section will introduce the participants and their background.

\begin{itemize}

    \item \textbf{Ronja} is 22 years old, and she ranks at 800 ELO, which fits in the target users of beginners.
    She was taught to play in her kindergarten, and now she plays chess occasionally for fun.

    \item \textbf{Clara} is 19 years old, and she ranks at 1000 ELO, which also fits in the target users.
    She studies machine learning, so she could provide valuable feedback on the Stockfish implementation.
    She was taught to play chess in an academic setting.

    \item \textbf{Peter} is 31 years old, and his rank is 1700 ELO, which is above the target users.
    He plays chess semi-professionally, first learning to play from his father and then in school.
    % textidote: ignore begin % due to chess.com.
    He got more interested in chess during the pandemic due to Chess.com's popularity and chess' presence on Twitch.
    % textidote: ignore end
    While he may not be the target users range, his experience could provide the team with valuable feedback.

\end{itemize}

% textidote: ignore begin % due to section being too short
\section{Results}\label{sec:results}
% textidote: ignore end

The results of the test are based on the feedback from the participants.
It is compacted into two main focus points which are the learning aspect of the project and a potential solution to
their feedback.
The last subsection will cover any other feedback that the participants provided.

\subsection{Learning experience}\label{subsec:learning-experience}

The main focus point with the usability testing was the learning aspect of the project, because that's the goal of the
project.
The feedback on the learning aspect was not very positive.
Ronja didn't immediately understand what the Stockfish feedback was, but after using Stockfish a little, she lost
interest in it.
Clara thought that the Stockfish feedback was too direct, and Peter mentioned that the feedback should provide future
moves along with the current move.
None of them thought that the feedback was helpful.
This indicates that the feedback from Stockfish has a lot of room for improvement, as only providing the best move is
not enough for learning, because players wouldn't understand why it's the best move.
The team was already aware of this issue and had discussed a potential solution, which was conveyed to the users in the
next focus point.

\subsection{Potential solution}\label{subsec:potential-solution}

As a potential solution, users were asked about the idea of using a large language model along with Stockfish to provide
feedback.
The rationale behind this question was due to a prior discussion within the team, which will be later discussed in
Section~\ref{sec:large-language-model-implementation}.
Clara and Ronja both liked the idea of using a large language model, while Peter was more skeptical.
He did not particularly like the idea due to the lack of control the developers would have over the large language
model's answers.
This will make it such that the team cannot be guaranteed that it will provide the correct information.
Clara, however, thinks that in combination with Stockfish it could provide accurate information whilst also giving more
natural responses and explanations.
She and Ronja think that a natural response would help a lot with understanding the feedback.
An example given was that the player could ask follow-up questions, which the large language model could answer,
providing a more interactive learning experience.

\subsection{Other feedback}\label{subsec:other-feedback}

The team got a lot of feedback unrelated to Stockfish that can also hinder the learning experience.
Peter pointed out that the front page was not intuitive, and Clara suggested that the buttons should be larger, labeled,
and link together.
Peter also mentioned that it was challenging to figure out how to play, saying that the About page should contain
more technical information.
Clara noted that the Stockfish feedback lacked visual cues, which is something that the team had also previously
discussed.
Peter mentioned that the notations provided by Stockfish were overcomplicated, complained about the lack of sound
effects and no clear indication for when a game is over.
Clara mentioned that the game should have more teaching tools, such as puzzles and guides to how to play chess.
The team considered these features given the learning nature of the game, but they focused on the coaching aspect of the
game instead.

\section{Conclusion}\label{sec:tests-conclusion}

While there are a lot of issues that need to be addressed, the overall opinion of the project was positive.
At the end of the usability testing, the participants were asked for their final thoughts on the project.
Clara and Ronja had a positive experience, and they would use the website again because they like the idea of using an
engine like Stockfish to improve their chess skills.
Peter, on the other hand, would rather use alternatives like~\url{Chess.com} and Lichess, because he believes that they
can do everything our website can do and more.
The main point of concern was the Stockfish feedback, because that's the main feature of the project.
The team has already discussed a potential solution to this issue, which was well received by the participants.
There was also a lot of great feedback that the team was not aware of, which will be taken into consideration when
improving the website and game.
However, since the focus of the project is on the learning aspect, the main priority will be to improve the Stockfish
feedback.
Two of the three participants like the idea of using a large language model to provide feedback, so the team is more
confident in this solution.
Overall, the usability testing was a success, and the feedback provided valuable insights for the team on how to improve
the website and game before the final release.

    % textidote: ignore begin
\chapter{Discussion}\label{ch:discussion}
% textidote: ignore end

% textidote: ignore begin
\section{Software requirements evaluation}\label{sec:software-requirements-evaluation}
% textidote: ignore end

This section refers back to Section~\ref{sec:software-requirement-specification} and more specifically
Table~\ref{tab:srs}.

% textidote: ignore begin

\subsection{Must-haves}\label{subsec:must-haves}
% textidote: ignore begin

This section concerns the evaluation of the must-have requirements.
Most must-have requirements are satisfied in the development of the first release of the ChessTeacher project.
This entails the basic chess functionality, the online multiplayer functionality and the chess engine implementation.
They work seamlessly and efficiently and are well integrated into the overall project.

% textidote: ignore begin

\subsubsection{Applied learning principles}
% textidote: ignore begin

Though most features are well implemented as mentioned above, there are some missing features.
The project is missing some key features regarding the learning principles mentioned in
Section~\ref{sec:learning-methods-for-chess}.
Key missing features include explanations and interactivity.
As chess as a game is complex, there are too many different positions for it to make sense to hard-code responses for
different positions.
To reach the full potential of the idea, the project needs some form of dynamic language model that is able to interpret
the Stockfish provided feedback, process this information and provide tailored feedback to the user.
This implementation is further discussed in Section~\ref{sec:large-language-model-implementation}.
These key missing features make for worse initial feedback by the test subjects as further discussed in
Section~\ref{sec:results}.
Even though this is undesired feedback and functionality for the overall potential of a chess learning web
application, the developers are still satisfied with the result.

% textidote: ignore begin

\subsubsection{Complex chess rules}
% textidote: ignore begin

Another feature that has not been included is the time control rule of chess.
Chess is usually played with time restrictions, no matter which variant of chess is in question, i.e.\ international,
blitz, rapid, bullet, etc.
Even though time control is a vital game mechanic in chess both competitively and on an amateur level it has not been
implemented, as the purpose of the web application is to learn chess rather than to play chess.
Individuals may dislike a timer counting them down whilst trying to learn chess, which is an argument for not
implementing the feature.
However, it is also important to learn where and when in the game to spend one's time thinking; why one could argue that
it should have been implemented to support learning of time management in chess.

% textidote: ignore begin

\subsection{Should-haves}\label{subsec:should-haves}
% textidote: ignore begin

This section concerns the evaluation of the should-have requirements.
Once again, most should-have requirements are satisfied in the development of the first release of the ChessTeacher
project.
This entails the implementation of user accounts, user registration and authentication, chat functionality and game
listing.
As the previous requirements, the mentioned factors all work efficiently and smoothly,
as also supported by the usability tests.

% textidote: ignore begin

\subsubsection{Engine feedback in English}
% textidote: ignore begin

However, the english feedback from Stockfish is, as per the first release of the project, hard-coded to be a random
message from a selection of three constructed sentences.
This is due to the same reasons as mentioned in Section~\ref{subsec:must-haves}.
Even though the implementation of the english user feedback is suboptimal, it still functions as desired when accounting
for the time and experience limitations.

Another implementation not prioritized due to time constraints is the front-end implementation of the
\verb|analyze_move| function mentioned in Section~\ref{subsec:stockfish-image-functions}.
Implementing the functionality would serve even more feedback to the user about their move.
This can help them improve and learn as mentioned in Section~\ref{subsec:assisted-learning}.

% textidote: ignore begin
\section{Large language model implementation}\label{sec:large-language-model-implementation}
% textidote: ignore end

As mentioned in Section~\ref{subsec:must-haves}, some key features, hereby explanations and interactivity, are not
implemented in the first release of the ChessTeacher project.
% textidote: ignore begin % [lt:en:EN_REPEATEDWORDS]
This makes for an application that in the current state may not suffice to resolve the problem statement.
There are several reasons for this gap between our solution and a solution that sufficiently satisfies the problem.
% textidote: ignore end

One reason is, that adding the necessary missing features is rather complex and time-consuming.
As the project has a strict deadline and as the developers are learning the languages and software whilst making the
project, time is somewhat limited.
Prioritizing the features to implement is therefore necessary.
One could argue that the missing features are more vital to the problem than some not as vital implemented features
and should therefore be prioritized higher.
Even though this is true, the reason is that the developers did not have a finished plan for the project at the
beginning of the project and therefore did not have a completed prioritized feature list as a baseline.
Throughout the development process, features are added and removed as the developers' knowledge-base around the topic of
the project improves and grows.
It is therefore difficult to start from a completed features list.
Another argument is that the developers are implementing features chronologically, why some less important features are
implemented before features that has a higher priority.
This approach may not be the most effective in the long term, however, it provides a simpler way to better visualize the
overall progress.

Another reason is, that the developers do not have the experience yet to know how to effectively implement the features.
Even though the process of implementing the core of LLM's is not too complicated, there are a lot of pitfalls that on
the contrary are complicated and are not worth the risk to potentially encounter and having to solve.
An example of a pitfall is that LLM API's most often have a price tag, and often a rather steep one.
It is not straightforward how do deal with limiting the amount of requests made to an LLM API and how to avoid other
common pitfalls, as the developer team is rather new to the field of software engineering and web development
especially.
The potential consequences of overlooked flaws in the code resulting in overdraft or other financial pitfalls is
determined to be too great for it to be worth implementing in the first release of the program.

Therefore, due to the reasons mentioned above LLM implementation is ultimately excluded.
One potential future solution to implement these missing features could be to implement them using proprietary software
such as ChatGPT to generate the feedback based on the Stockfish evaluation.


    % textidote: ignore begin
\section{Conclusion}\label{sec:conclusion}
% textidote: ignore end

The group has taken feedback from the previous process analysis and is confident that they've improved their teamwork
since P1.
However, there are still some areas that could be improved, but they are far fewer than in the previous project.
Furthermore, the group has learned a lot of new tools that could be used in future projects.
Therefore, they are satisfied with the work they have done in this project.


    % Appendix
    \appendix

    % textidote: ignore begin
\chapter{Source Code Repositories}\label{ch:source-code-repositories}
% textidote: ignore end

Reference is made to our GitHub repositories:

\begin{itemize}
    \item \href{https://github.com/audio-engineer/aau-p2-article}{Article repository}, release 1.0.0.
    Link: \url{https://github.com/audio-engineer/aau-p2-article}.
    \item \href{https://github.com/audio-engineer/chess-teacher-firebase}{Firebase image repository}, release 1.0.0.
    Link: \url{https://github.com/audio-engineer/chess-teacher-firebase}.
    \item \href{https://github.com/audio-engineer/chess-teacher-front-end}{Frontend repository} release, 1.0.0.
    Link: \url{https://github.com/audio-engineer/chess-teacher-front-end}.
    \item \href{https://github.com/audio-engineer/chess-teacher-stockfish}{Stockfish image repository} release, 1.0.0.
    Link: \url{https://github.com/audio-engineer/chess-teacher-stockfish}.
\end{itemize}

    \subfile{process-analysis/process-analysis}

    % Bibliography
    \printbibliography[heading=bibintoc]
\end{document}
