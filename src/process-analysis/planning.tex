% textidote: ignore begin
\section{Planning}\label{sec:planning}
% textidote: ignore end

The main issue when working on the project has been bad oversight
into how much work goes into specific parts of developing a product.
In the P1 project, the main problem was lack of time for improving the code
and implementing fixes that were found in the wake of the release.

For the P2 project, the group decided to introduce scrum as a tool to
better plan work, as there was a need to combat pushing most of the
work too close to the deadline.

Scrum is a framework for working in an agile environment~\cite{agile-manifesto}.
To fully implement Scrum would be overkill for this project, and the authors
therefore decided to cut back on some of the more stringent requirements of Scrum.
The implementation used during the P2 project involved weekly virtual or physical meetings where each team member
presented what they had been working on.
This also functioned as a good place to ask for assistance or other questions the members had.

To get a better overview of all the work that should be done, the authors made use of GitHub
Issues~\cite{github-issues}.
This solved most problems in regard to task management, but did require some discipline to consistently create and check
issues that needed to be solved.
As this was not a problem with the GitHub Issues, but more self-control of the members, we solved this by helping each
other use the tools correctly, which made adoption easier.

Major deadlines were managed by the team members, but because of the authors want for less restriction, the minor
deadlines were not well planned or discussed.
This resulted in the supervisor stepping in if it seemed that the project was too slow in certain areas.
That is not to say that things did not get worked on without intervention from the supervisor.
This is because the team had experience with self-assigning tasks and contributing when necessary.
This meant that they were not behind by much, but that there were smaller issues with backlogging tasks that were not
interesting to do or write.

During the project, back casting was implemented to have control over bigger deadlines, such as status seminars and the
turn-in deadline for the project.
This helped make sure everything was turned in at the correct time, and with the help of the supervisor, minor goals
were also set to make the road to the end goal more concrete.
