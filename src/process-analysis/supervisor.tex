% textidote: ignore begin
\section{Working with the supervisor}\label{sec:working-with-the-supervisor}
% textidote: ignore end

The group's supervisor was knowledgeable enough in regard to the technologies and methods the group used.
The supervisor assessed his task as a Laissez faire supervisor~\cite{PBL}.
This approach to supervision was effective for several reasons:

\begin{itemize}
    \item The group was able to develop own initiatives and realize them as needed, which made for greater authenticity
    in both the software and the report.
    \item The group had greater control over the process in regard to both decision-making and milestones.
    \item The group had the opportunity to opt out of the standard pr\. week meeting when it was not necessary, which
    made for better time efficiency in periods of greater production.
    \item The group had a greater number of responsibilities, which made for a greater learning experience.
\end{itemize}
