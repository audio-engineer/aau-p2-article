% textidote: ignore begin
\section{Learning}\label{sec:learning}
% textidote: ignore end

The group comprises five of the students from the P1 project.
Therefore, the students in the group are familiar with each other, both at a personal and an academic level.
As per P1, the P2 project tech-stack is rather complicated and out of scope.
However, as the group is very ambitious and wants to learn relevant technologies, this complexity is welcomed.

Some technologies the group learned throughout the project are:

\begin{itemize}
    \item \textbf{Docker}: Docker is vital for the development environment for all the members of the group, as all the
    three main operating systems, Windows, macOS and Linux, are represented in the group.
    Learning this tool made for a smoother development process and probably for significantly less time debugging.
    \item \textbf{TypeScript}: As JavaScript is a rather complicated programming language that has a lot of unique
    behavior, especially from the eyes of beginners, the group learned TypeScript, as it provided better and more
    type safe code.
    \item \textbf{React}: As HTML, CSS and JavaScript are rather bare-bones, it takes a lot of time to code beautifully
    and efficiently using just these technologies, why React had to be learned.
    React made for more efficient development and for better readability, which helped the group better communicate
    about the code and better understand the code others have written.
    % textidote: ignore begin % due to next.js being identified as an url
    \item \textbf{Next.js}: Next.js provided the group with an easier way of routing by using the file system using the
    % textidote: ignore end
    app directory among other useful features.
    This framework once again made for more efficient code writing and understandability that the group benefited
    greatly from.
    \item \textbf{Python}: For the parsing of the Stockfish executable output a Python script was used, why the group
    had to learn the foundational elements of the Python programming language.
    Learning this technology was necessary for the project and will also most probably help the group members in their
    later career, why it was decided to learn it.
\end{itemize}
