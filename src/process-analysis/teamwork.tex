% textidote: ignore begin
\section{Teamwork in the group}\label{sec:teamwork-in-the-group}
% textidote: ignore end

The teamwork in the group has been similar to that of the P1 project.
As all five members of the P2 group were also a member of the P0 and P1 projects, they already knew each other.

There have been elements of the teamwork that went well and also elements that leave room for improvement.
Some of the processes that went well are mentioned below:

\begin{itemize}
    \item \textbf{Communication} has improved since P1, as members participate more in pull request reviews and idea
    generation.
    This could be because the group has five members in P2 compared to the seven of P1, which means there
    is more room for the individuals in the P2 group to communicate their ideas.
    \item \textbf{Report writing} has gone better compared to P1, as fewer segments are rewritten and less time is spent
    discussion the layout.
    This is probably because the team members are getting more and more used to the process of writing the reports each
    semester.
\end{itemize}

Some of the processes that leave room for improvement are mentioned below:

\begin{itemize}
    \item \textbf{Work load balancing} is an obstacle that the team is still solving.
    There are clear periods in the code frequency for the two main repositories where productivity is high and low
    respectively.
    Some of the discrepancy is because the problem analysis is written before the software, which in turn
    is written before the problem solution.
    However, there are still periods where comparatively low and high amounts of productivity are present.
    There will always be periods with different amount of productivity; however, the fluctuations in the group's code
    frequency can be smoothed out.
    \item \textbf{Work load distribution} is the challenge of making sure that all members are doing an equal amount
    of work in relation to the project.
    In P2, some members of the team have done more work than others on the software as well as the report.
    This can be due to many different reasons.
    One reason could be that some people are more comfortable with the used technologies and therefore are more
    efficient in their work and thereby producing more.
\end{itemize}
